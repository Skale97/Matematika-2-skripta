\documentclass[a4paper, titlepage, 10pt]{article}

\usepackage[margin=1in]{geometry}
\usepackage[utf8]{inputenc}
\usepackage[T1]{fontenc}
\usepackage[serbian]{babel}
\usepackage{mathtools, amssymb, amsfonts, amsthm}
\usepackage{tabu}
\usepackage{centernot}
\usepackage{enumitem}

\author{Marko Skakun}
\title{\textbf{\LARGE MATEMATIKA 2} \\ \normalsize \textit{Skripta}}

\newtheorem{theorem}{Teorema}
\newtheorem{corollary}{Posledica}[theorem]
\newtheorem{definition}{Definicija}
\newtheorem{example}{Primer}

\newcommand{\card}{\ensuremath{{\rm card\ }}}
\newcommand{\realset}{\mathbb{R}}
\newcommand{\dif}{\, \mathrm{d}}
\newcommand{\powerset}[1]{\mathbb{P}(#1)}
\newcommand{\Napomena}{\textbf{Napomena: }}

\everymath{\displaystyle}

%Resetovanje brojača teorema po poglavlju%
\makeatletter
\@addtoreset{theorem}{section}
\@addtoreset{definition}{section}
\@addtoreset{corollary}{section}
\@addtoreset{example}{section}
\makeatother

\let\oldemptyset\emptyset
\let\emptyset\varnothing

\begin{document}
	\let\thefootnote\relax\footnotetext{\textit{Napomena:} u svesci trenutno nisu ubačeni primeri jer se veoma sporo kucaju, ali je sva teorija proverena za sada dva puta. U slučaju da pronađete neku grešku (slovnu, konceptualnu,...), ili imate neku preporuku pošaljite je na mail: skalesms@gmail.com}
	\maketitle
	\newpage
	\tableofcontents
	\newpage
	
\section{Neodređeni integrali} 
\subsection{Uvod} 

%Primitivna funkcija
\begin{definition}
	Neka je funkcija $f$ definisana na proizvoljnom intervalu $I = (a, b) \quad a,b\in \overline{\mathbb{R}}$. Ako postoji funkcija $F(x)$ koja je definisana na $I$ i za koju važi $$(\forall x \in I)\; F'(x) = f(x)$$ tada kažemo da je $F(x)$ \textbf{primitivna funkcija} funkcije $f(x)$ na intervalu $I$.
\end{definition}

\begin{theorem}
	Ako je $F(x)$ primitivna funkcija funkcije $f$ na intervalu $I$, tada je i $F(x)+c; \quad c \in \mathbb{R}$ takođe primitivna funkcija funkcije $f$ na intervalu $I$.
\end{theorem}

\begin{proof}
	$$(\forall x \in I)(F(x)+c)' = F'(x) = f(x)$$
\end{proof}

\begin{theorem}
	Ako su $F_{1}(x)$ i $F_{2}(x)$ neke dve primitivne funkcije funkcije $f$ na intervalu $I$, tada:
	$$(\exists c \in \mathbb{R})(\forall x \in I)\;F_{1}(x)-F_{2}(x) = c$$
\end{theorem}

\begin{proof}
	\begin{align*}
		&\varphi (x) = F_{1}(x) - F_{2}(x)\\
		&\varphi'(x) = F_{1}'(x) - F_{2}'(x) = f(x)-f(x) = 0\\
		&(\forall x \in \mathbb{R}) \;\varphi'(x) = 0 \implies \varphi(x) = c\
	\end{align*}
	* Teorema iz diferencijalnog računa $(\exists c \in \mathbb{R})(\varphi(x) = c) $
\end{proof}

\begin{theorem}
	Ako je funkcija neprekidna na intervalu $I$ tada ona ima primitivnu funkciju na intervalu $I$.\\
\end{theorem}

%Neodređeni integral
\begin{definition}
	Skup svih primitivnih funkcija funkcije $f$ na intervalu $I$ naziva se \textbf{neodređeni integral} funkcije $f$ na intervalu $I$.\\
	Oznaka:
	$$ \int f(x) \, \mathrm{d}x; \quad \int f(x)\, \mathrm{d}x= \{F(x)+c\;|\;c\in\mathbb{R}\}$$
	Po dogovoru: $\int f(x)\, \mathrm{d}x = F(x)+c; \quad c \in \mathbb{R}$ 
\end{definition}

\begin{theorem}
	Neka funkcije $f$ i $g$ imaju primitivne funkcije na intervalu $I$, i neka je\\ $\int f(x)\, \mathrm{d}x = F(x) + c; \quad c \in \mathbb{R}$, tada važi:
	\begin{enumerate}
		\item $\mathrm{d}\left(\int f(x)\, \mathrm{d}x\right) = f(x)dx$
		\item $\left(\int f(x)\, \mathrm{d}x\right)'= \frac{d\left(\int f(x)\, \mathrm{d}x\right)}{dx} = f(x)$
		\item $\int \mathrm{d}F(x) = F(x) + c$
	\end{enumerate}
\end{theorem}

\begin{proof}
	\begin{enumerate}
		\item $d\left(\int f(x)\, \mathrm{d}x \right) = d(F(x) + c) = dF(x) = F'(x)dx = f(x)dx$
		\item $\left(\int f(x)\, \mathrm{d}x\right)' = f(x)$
		\item $\int \mathrm{d}F(x) = \int F'(x)\, \mathrm{d}x = \int f(x) \, \mathrm{d}x = F(x) + c$
	\end{enumerate}
\end{proof}


\subsection{Osnovni metodi integracije}

\subsubsection{Tablica}
	Tablica neodređenih integrala
	\begin{enumerate}[label = \arabic*)]
		\item $\int x^n\, \mathrm{d}x = \frac{x^{n+1}}{n+1}+c; \quad n \centernot = -1$
		\item $\int \frac{1}{x}\, \mathrm{d}x = \ln |x| + c; \quad x \centernot = 0$
		\item $\int a^x \, \mathrm{d}x = \frac{a^x}{\ln a} + c; \quad a>0, a \centernot = 1$
		\item $\int e^x \, \mathrm{d}x = e^x + c$
		\item $\int \frac{\mathrm{d}x}{1+x^2} = \arctg x + c = - \arcctg x + c_1$
		\item $\int \frac{\mathrm{d}x}{\sqrt{1 - x^2}} = \arcsin x + c = - \arccos x + c_1; \quad (|x| < 1)$
		\item $\int \frac{\mathrm{d}x}{\sqrt{x^2 \pm 1}} = \ln |x+\sqrt{x^2 \pm 1}| + c = 
		\begin{cases}
			\arsh x + c \quad za \; +\\
			\arch x + c \quad za - i \; x > 1
		\end{cases}$
		\item $\int \frac{\, \mathrm{d}x}{1-x^2} = \frac{1}{2}\ln \left | \frac{1 + x}{1 - x} \right | + c = 
		\begin{cases}
			\arth x + c &\quad za |x|<1\\
			\arcth x + c &\quad za |x|> 1
		\end{cases}$
		\item $\int \sin x \, \mathrm{d}x = - \cos x + c$
		\item $\int \cos x \, \mathrm{d}x = \sin x + c$
		\item $\int \frac{\mathrm{d}x}{\cos^2 x} = \tg x + c$
		\item $\int \frac{\mathrm{d}x}{\sin^2 x} = - \ctg x + c$
		\item $\int \sh x \, \mathrm{d}x = \ch x + c$
		\item $\int \ch x \, \mathrm{d}x = \sh x + c$
		\item $\int \frac{\mathrm{d}x}{\ch^2 x} \th x + c$
		\item $\int \frac{\mathrm{d}x}{\sh^2 x} = -\cth x + c$
	\end{enumerate}
	Smenom izvedeni neodređeni integrali
	\begin{enumerate}[label = \arabic*)]
		\item $\int \frac{\mathrm{d}x}{\sqrt{x^2 \pm a}} = \ln |x + \sqrt{x^2 \pm a}| + c$
		\item $\int \frac{\mathrm{d}x}{\sqrt{a^2 - x^2}} = \arcsin \frac{x}{a} + c$
		\item $\int \frac{\mathrm{d}x}{x^2+a^2} = \frac{1}{a}\arctan\frac{x}{a} + c$
		\item $\int \frac{\mathrm{d}x}{x^2-a^2} = \frac{1}{2a}\ln\left | \frac{x-a}{x+a} \right | + c$
		\item $\int \frac{\mathrm{d}x}{a^2 - x^2} = \frac{1}{2a} \ln \left | \frac{a+x}{a-x} \right | + c$
	\end{enumerate}
	
	
	
\subsubsection{Korišćenjem teoreme o linearnosti integrala}
	\begin{theorem}
		Ako funkcije $f$ i $g$ imaju primitivne funkcije na intervalu $I$, i ako su $\alpha, \beta \in \mathbb{R}$, tada važi sledeće (na intervalu $I$):
		$$\int (\alpha f(x)+\beta g(x)) \, \mathrm{d}x = \alpha \int f(x)\, \mathrm{d}x + \beta \int g(x)\, \mathrm{d}x + c$$
	\end{theorem}
	\begin{proof}
		\begin{align*}
		(\forall x \in I) &\left(\int (\alpha f(x) + \beta g(x))\, \mathrm{d}x\right)'  = \alpha f(x) + \beta g(x) \\
		(\forall x \in I) &\left(\alpha \int f(x)\, \mathrm{d}x + \beta \int g(x)\, \mathrm{d}x\right)' = \alpha\left(\int f(x)\, \mathrm{d}x\right)' + \beta \left(\int g(x)\, \mathrm{d}x\right)' = \alpha f(x) + \beta g(x)\\
		\implies &\int (\alpha f(x)+\beta g(x)) \, \mathrm{d}x = \alpha \int f(x)\, \mathrm{d}x + \beta \int g(x)\, \mathrm{d}x + c\\
		\end{align*}
	\end{proof}
	
	
\subsubsection{Metod smene promenljive}
	\begin{theorem}
		Neka je data funkcija $f$ koja je neprekidna na intervalu $I$ i neka za funkciju $\varphi : I_1 \to I$ (gde je $I_1$ interval) važi:
		\begin{enumerate} 
			\item $\varphi$ i $\varphi'$ neprekidne na intervalu $I_1$
			\item $(\forall t \in I_1) \varphi' (t) \centernot = 0$
			\item $\exists \varphi^{-1}$
		\end{enumerate}		 
	 	Tada je $\int f(x) \, \mathrm{d}x = \int f(\varphi (t)) \varphi'(t)\, \mathrm{d}t$, smena $x = \varphi(t); \; \mathrm{d}x = \varphi'(t)\mathrm{d}t$
	\end{theorem}
	\begin{proof}
		$$\left(\int f(x)\, \mathrm{d}x \right)' = f(x) = L\\$$
		\begin{multline*}
			\left[
			\begin{aligned}
				x &= \varphi(t)\\
				dx &= \varphi'(t)dt
			\end{aligned}
			\right] \quad
			\left(\int f(\varphi (t))\varphi ' (t) \, \mathrm{d}t\right)'  
			=\frac{d\left(\int f(\varphi (t))\varphi'(t)\, \mathrm{d}t\right)}{dx} = \\
			=\frac{d\left(\int f(\varphi (t))\varphi'(t)\, \mathrm{d}t\right)}{dx}\frac{dx}{dt}\frac{dt}{dx} =
			f(\varphi (t))\varphi'(t)\frac{1}{\varphi'(t)} = 
			f(\varphi(t)) = f(x) = D\\
		\end{multline*}
		$$L = D$$
	\end{proof}
	
	\begin{corollary}
		\begin{enumerate}[label = \arabic*)]
			\item 
				Poznato je $\int f(x)\, \mathrm{d}x = F(x) + c$\\
				Traži se $\int f(\varphi (t))\varphi'(t)\, \mathrm{d}t$\\
				$x = \varphi(t)$\\
				$\int f(\varphi(t))\varphi'(t)\, \mathrm{d}t = F(\varphi(t))+c$
			\item 
				Poznato je $\int f(\varphi(t))\varphi'(t)\, \mathrm{d}t = G(t) + c_1$\\			
				Traži se $\int f(x)\, \mathrm{d}x$\\
				$x = \varphi(t)$\\
				$t = \varphi^{-1}(x)$\\
				$\int f(x)\, \mathrm{d}x = G(\varphi^{-1}(x))+c_1$
		\end{enumerate}
	\end{corollary}
	
\subsubsection{Svođenje kvadratnog trinoma na kanonski oblik}
	\begin{align*}
		ax^2+bx+c = a\left(x^2+\frac{b}{a}x+\frac{c}{a}\right) &= a\left(x^2 + \frac{2b}{2a}x + \left(\frac{b}{2a}\right)^2 - \left(\frac{b}{2a}\right)^2 + \frac{c}{a}\right) =\\
		&= a\left(\left(x+\frac{b}{2a}\right)^2-\frac{b^2}{4a^2}+\frac{c}{a}\right) =\\
		&= a\left(\left(x+\frac{b}{2a}\right)^2 + \frac{4ac-b^2}{4a}\right)\\
		\text{smena: } &x+\frac{b}{2a} = t
	\end{align*}
	Opšti slučaj:
	\begin{enumerate}[label = \arabic*)]
		\item $\int \frac{\mathrm{d}x}{ax^2+bx+c}$
		\item $\int \frac{Mx+N}{ax^2+bx+c}\, \mathrm{d}x$
		\item $\int \sqrt{ax^2+bx+c}\, \mathrm{d}x$
		\item $\int \frac{\mathrm{d}x}{\sqrt{ax^2+bx+c}}$
		\item $\int \frac{Mx+N}{\sqrt{ax^2+bx+c}}\, \mathrm{d}$
	\end{enumerate}
	
	
	
\subsubsection{Parcijalna integracija}
		\begin{theorem}
			Neka su funkcije $u(x)$ i $v(x)$ diferencijabilne na intervalu $I$, tada važi:
			$$\int u(x)\, \mathrm{d}v(x) = u(x)v(x)-\int v(x)\, \mathrm{d}u(x) + c$$
		\end{theorem}
		\begin{proof}
			\begin{align*}
				&\mathrm{d}(u(x)v(x))=\mathrm{d}u(x)v(x)+u(x)\mathrm{d}v(x) \\
				&\int \, \mathrm{d}(u(x)v(x)) = \int \, \mathrm{d}u(x)v(x)+ \int u(x)\, \mathrm{d}v(x)\\
				&u(x)v(x) - \int v(x)\, \mathrm{d}u(x) = \int u(x)\, \mathrm{d}v(x)
			\end{align*}
		\end{proof}
		

\subsubsection{Metod rekurentnih formula}
	\begin{align*}
		I(n) &= \int f(x, n)\, \mathrm{d}x\\
		I(n-1) &= \int f(x, n-1)\, \mathrm{d}x\\
		I(n-2) &= \int f(x, n-2)\, \mathrm{d}x\\
	\end{align*}
	\begin{example}
		\begin{align*}
			I(n) &= \int \frac{1}{(1+x^2)^n}\, \mathrm{d}x\\
			n=1\; : \; I(n) &= \int \frac{1}{1+x^2}\, \mathrm{d}x = \arctg x + c\\
			n=2\; : \; I(n) &= ?\\
			n=3\; : \; I(n) &= ?\\
			\\
			I(n) &= \int \frac{1+x^2-x^2}{(1+x^2)^n}\, \mathrm{d}x =\\
			&=\int \frac{1}{(1+x^2)^{n-1}}\mathrm{d}x-\int \frac{xx}{(1+x^2)^n}\, \mathrm{d}x =\\
			&= \left[
			\begin{aligned}
				u(x) = x &\quad du = dx\\
				dv(x) = \frac{x}{(1+x^2)^n}dx &\quad v = \int \frac{x}{(1+x^2)^n}\, \mathrm{d}x = \frac{1}{2(1-n)}\frac{1}{(1+x^2)^{n-1}}			
			\end{aligned}
			\right]=\\
			&= \frac{-x}{(1+x^2)^{n-1}}\frac{1}{2(1-n)}+\frac{1}{2(1-n)}\underbrace{\int \frac{1}{(1+x^2)^{n-1}}\, \mathrm{d}x}_{I(n-1)} + \underbrace{\int \frac{1}{(1+x^2)^{n-1}}\, \mathrm{d}x}_{I(n-1)} =\\
			I(n) &= I(n-1)\left( 1+\frac{1}{2(1-n)}\right) - \frac{x}{2(1-n)(1+x^2)^{n-1}}\\
			n=1\; : \; I(1) &= \arctg x +c\\
			n=2\; : \; I(2) &= \arctg x\left(1+\frac{1}{2(1-2)}\right)-\frac{x}{2(1-2)(1+x^2)^{1-2}} = \frac{-\arctg x}{2} + \frac{x(1+x^2)}{2} +c
		\end{align*}
	\end{example}


\subsubsection{Metod neodređenih koeficijenata}
	Koristi se kada možemo da pretpostavimo analitički oblik primitivne funkcije
	\begin{example}
		$$\left. \int x e^x\, \mathrm{d}x = (Ax+B)e^x+c \middle/ \frac{d}{dx} \right.$$
		-Polinom prvog stepena sa neodređenim koeficijentima A i B
		\begin{gather*}
			xe^x = A e^x + (Ax+B)e^x\\
			xe^x = (Ax+A+B)e^x\\
			A=1 \; \wedge  \; A+B = 0\\
			A=1 \; \wedge \; B=-A=-1\\
			(A, B) = (1, -1) \implies \int xe^x \, \mathrm{d}x= (x-1)e^x +c
		\end{gather*}
	\end{example}
	\begin{example}
		\begin{gather*}
			\left. \int x^2 e^x \, \mathrm{d}x = (Ax^2+Bx+C)e^x +c \middle/ \frac{d}{dx} \right.\\
			x^2 e^x = (Ax^2+Bx+C+2Ax+B)e^x\\
			x^2 e^x = (Ax^2+(2A+B)x+B+C)e^x\\
			A = 1 \; \wedge \; 2A+B = 0 \; \wedge \;  B+C = 0\\
			A = 1 \; \wedge \; B = -2A = -2 \; \wedge \; C = -B = 2\\
			(A, B, C) = (1, -2, 2) \implies \int x^2 e^x \, \mathrm{d}x = (x^2-2x+2)e^x+c
		\end{gather*}
	\end{example}

\subsection{Posebne metode integracije}

\subsubsection{Integracija racionalnih funkcija}
	$$\int R(x)\, \mathrm{d}x = ?\\$$
	$R(x) = \frac{P_n (x)}{Q_m (x)}\quad dg P_n = n \geq dg Q_m = m \implies R(x)$ je \textbf{neprava} racionalna funkcija,\\
	za $n<m \implies R(x)$ je \textbf{prava} racionalna funkcija.\\
	Svaka neprava racionalna funkcija može da se prikaže kao zbir polinoma i prave racionalne funkcije:
	\begin{gather*}
		R(x) = T(x)+\frac{S_l(x)}{Q_m(x)}\quad dg S_l = l < dg Q_m = m\\
		\int R(x)\, \mathrm{d}x = \int T(x)\, \mathrm{d}x + \int \frac{S_l(x)}{Q_m(x)} \, \mathrm{d}x
	\end{gather*}
	\begin{theorem}
		Svaka prava, nesvodljiva racionalna funkcija može se na jedinstven način razložiti na zbir parcijalnih razlomaka.
		$$\frac{S_l(x)}{Q_m(x)} = \frac{1}{a_m}\left( \sum^k_{i=1} \sum^{r_i}_{j=1} \frac{A_{ij}}{(x-x_i)^j}+\sum^m_{i=1} \sum^{s_i}_{j=1} \frac{B_{ij}x+C_{ij}}{(x^2+p_ix+q_i)^j}\right)$$
		gde je $Q_m (x)=a_m(x-x_1)^{r_1}(x-x_2)^{r_2}\ldots(x-x_k)^{r_k}(x^2+p_1x+q_1)^{s_1}(x^2+p_2x+q_2)^{s_2} \ldots (x^2+p_mx+q_m)^{s_m}$
		-Opšti oblik parcijalnih razlomaka:
		$$\frac{A}{(x-a)^k};\; \frac{Bx+C}{(x^2+px+q)^m};\; k,m \in \mathbb{N};\; p^2-4q<0$$
	\end{theorem}
	
\subsubsection{Integracija nekih iracionalnih funkcija}
	\begin{enumerate}[label = \textbf{\arabic*.)}]
		\item 
			$\int R\left(x, \left(\frac{ax+b}{cx+d}\right)^{\frac{p_1}{q_1}}, \left(\frac{ax+b}{cx+d}\right)^{\frac{p_2}{q_2}}, \ldots, \left(\frac{ax+b}{cx+d}\right)^{\frac{p_n}{q_n}}\right)\, \mathrm{d}x$
			gde je $ad-bc \centernot = 0$\\
			- Ovaj integral se svodi na integral racionalne funkcije smenom:
			$$\frac{ax+b}{cx+d} = t^{NZS(q_1, q_2, \ldots, q_n)}$$
		\item 
			Ojlerove smene: $\int R\left(x, \sqrt{ax^2+bx+c}\right)\, \mathrm{d}x$ \\
			- Ovaj integral se svodi na integral racionalne funkcije smenom:
			\begin{enumerate}[label=\Roman*]
				\item
					Ojlerova smena:
					$$a>0 \implies \sqrt{ax^2+bx+c}=t\pm\sqrt{a}x$$
				\item
					Ojlerova smena:
					$$c>0 \implies \sqrt{ax^2+bx+c}=xt\pm\sqrt{c}$$
				\item
					Ojlerova smena:
					$$(x_1, x_2 \in \mathbb{R} \land x_1\centernot = x_2) \implies \sqrt{ax^2+bx+c} = t(x-x_1) \lor t(x-x_2)$$
			\end{enumerate}
			\textbf{Napomena:} Ojlerove smene treba izbegavati kada je god to moguće jer dovode do integrala racionalnih funkcija koji nisu baš zgodni za rešavanje
		\item
			Integracija binomnog diferencijala $\int x^m(a+bx^n)^p\, \mathrm{d}x$ \\ 
			gde su $a, b\in \mathbb{R};\; m, n, p \in \mathbb{Q};\; m = \frac{m_1}{m_2};\; n = \frac{n_1}{n_2};\; p = \frac{p_1}{p_2}; \; ab \centernot = 0;\; np\centernot = 0$\\
			- Rešenje se može predstaviti elementarnim funkcijama (i to svođenjem na integrale racionalnih funkcija) \textbf{samo} u sledećim slučajevima
			\begin{enumerate}[label = \Roman*]
				\item 
					Ako je $p \in \mathbb{Z}$
					\begin{enumerate}[label = \alph*)]
						\item za $p>0$ razlaganje po binomnoj formuli svodi se na zbir tabličnih integrala
						\item za $p<0$ koristi se smena $x = t^{NZS(m_2, n_2)}$
					\end{enumerate}
				\item
					Ako je $\frac{m+1}{n}\in \mathbb{Z}$ koristi se smena  $a+bx^n=t^{p_2}$
				\item
					Ako je $\frac{m+1}{n}+p \in \mathbb{Z}$ koristi se smena $ax^{-n}+b=t^{p_2}$
			\end{enumerate}
		\item
			U integrale koji sadrže $\sqrt{a^2-x^2}$, $\sqrt{x^2-a^2}$, $\sqrt{a^2+x^2}$ često je pogodno uvesti trigonometrijsku smenu:
			\begin{enumerate}[label = \alph*)]
				\item 
					$\sqrt{a^2-x^2}$ uvodi se smena $x = a \sin t\quad dx = a \cos t dt \quad t = \arcsin \frac{x}{a}$
				\item
					$\sqrt{x^2+a^2}$ uvodi se smena $x = a \tg t \quad dx = \frac{a}{\cos^2 t}dt \quad t = \arctg \frac{x}{a}$
				\item 
					$\sqrt{x^2-a^2}$ uvodi se smena $x = \frac{a}{\cos t} \quad dx = \frac{a \sin t}{\cos^2 t}dt \quad t = \arccos \frac{a}{x}$
			\end{enumerate}
	\end{enumerate}

	\textbf{Napomena:} \begin{gather*}
		\int e^{-x^2}\, \mathrm{d}x; \; \int x^{2n}e^{\pm x^2}\, \mathrm{d}x; \; \int \cos (x^2) \, \mathrm{d}x; \; \int \sin(x^2) \, \mathrm{d}x \\
 		\int \frac{\sin x}{x^n} \, \mathrm{d}x; \; \int \frac{\cos x}{x^n}\, \mathrm{d}x;/; \int \frac{e^ax}{x^n}\, \mathrm{d}x ;\; \int \frac{x^n}{\ln x}\, \mathrm{d}x;\; \ldots
	\end{gather*}\\
	Ovo su neke od funkcija koje se ne mogu izraziti preko elementarnih funkcija.
	
\subsubsection{Integracija trigonometrijskih funkcija}
	Integral oblika $\int R (\sin x, \cos x)\, \mathrm{d}x$ gde je $R(\sin x, \cos x)$ racionalna funkcija po $\sin x$ i $\cos x$ svodi se na integral racinalne funkcije na sledeći način:
	\begin{enumerate}[label=\textbf{\arabic*.)}]
		\item 
			Ako je $R (-\sin x, -\cos x) = R (\sin x, \cos x)$ uvodi se smena 
			\begin{align*}
				\begin{aligned}
					t &= \tg x\\
					dx &= \frac{dt}{1+t^2}\\
				\end{aligned}
				\quad \quad
				\begin{aligned}
					\cos x &= \frac{1}{\sqrt{1+t^2}}\\
					\sin x &= \frac{t}{\sqrt{1+t^2}}
				\end{aligned}
			\end{align*}
		\item 
			Ako je $R (+\sin x, -\cos x) = -R (\sin x, \cos x)$ uvodi se smena 
			\begin{align*}
				\begin{aligned}
					t &= \sin x\\
					dx &= \frac{dt}{\sqrt{1-t^2}}\\
				\end{aligned}
				\quad \quad
				\begin{aligned}
					\cos x &= \sqrt{1-t^2}\\
				\end{aligned}
			\end{align*}
		\item 
			Ako je $R (-\sin x, +\cos x) = -R (\sin x, \cos x)$ uvodi se smena 
			\begin{align*}
				\begin{aligned}
					t &= \cos x\\
					dx &= -\frac{dt}{\sqrt{1-t^2}}\\
				\end{aligned}
				\quad \quad
				\begin{aligned}
					\sin x &= \sqrt{1-t^2}\\
				\end{aligned}
			\end{align*}
		\item Univerzalna trigonometrijska smena može se uvek primeniti
			\begin{align*}
				\begin{aligned}
					t &= \tg \frac{x}{2}\\
					dx &= \frac{2dt}{1+t^2}\\
				\end{aligned}
				\quad \quad
				\begin{aligned}
					\sin x &= \frac{2t}{1+t^2}\\
					\cos x &= \frac{1-t^2}{1+t^2}
				\end{aligned}
			\end{align*}
			-\textbf{Napomena:} izbegavati univerzalnu trigonometrijsku smenu.
	\end{enumerate}
	\section{Određeni integrali}
\subsection{Uvod}
\begin{definition}
	Neka je data funkcija $f$ definisana na segmentu $[a, b]$ gde su $a, b \in \mathbb{R} \land a<b$. Uređena $m$-torka $d = (x_0, x_1, \ldots, x_n, \xi_0, \xi_1, \ldots, \xi_{n-1})$ takva da je $a \equiv x_0 < x_1 < \ldots < x_n \equiv b\; \land \;x_i \leq \xi_i \leq x_{i+1}; \; i=\overline{0, n-1}$ nazivamo \textbf{podelom} segmenta $[a, b]$
\end{definition}
\begin{definition}
	 Suma $\sum f(\xi_i)(x_{i+1}-x_i)$ naziva se \textbf{Rimanova (integralna) suma} funkcije $f$ na intervalu $[a, b]$ za datu podelu $d$.\\
	 \textit{Oznaka:} $$S(f, d, a, b)$$
\end{definition}
\begin{definition}
	\textbf{Norma} podele d je $||d|| = max(x_{i+1}-x_i); \; i=\overline{0,n-1}$.\\
	\textbf{Napomena:} Iz ove definicije sledi da ako $||d|| \to 0$ tada $(\forall i) x_{i+n}-x_i \to 0 \land n \to \infty$
\end{definition}
\begin{definition}
	Neka je data funkcija $f$ definisana na segmentu $[a, b]$ gde su $a, b \in \mathbb{R} \land a<b$. Ako postoji realan broj $I$ takav da važi:
	$$(\forall \varepsilon > 0)(\exists \delta(\varepsilon)>0)(\forall d)\left(||d||<\delta(\epsilon) \implies \left|I- \sum^{n-1}_{i = 0}f(\xi_i)(x_{i+1}-x_i) \right|<\varepsilon\right)$$
	$$\text{tj. } \lim_{||d||\to 0} S(f, d, a, b) = I$$ tada se broj $I$ naziva \textbf{određeni (Rimanov) integral} funkcije $f$ na segmentu $[a, b]$. \\
	\textit{Oznaka:} $$I = \int\limits^b_a f(x)\, \mathrm{d}x$$
\end{definition}
\begin{definition}
	Ako postoji broj $I\in \mathbb{R}$ takav da je $\lim_{||d||\to 0} S(f, d, a, b) = I$ kažemo da je funkcija $f$ \textbf{integrabilna} na segmentu $[a,b]$.
\end{definition}


\subsection{Potrebni i dovoljni uslovi za integrabilnost funkcija}

\begin{definition}
	(Darbuove sume funkcije $f$ na segentu $[a,b]$)\\
	Neka je funkcija $f$ ograničena na segmentu $[a,b]$. Ako segment $[a,b]$ podelimo tačkama $a \equiv x_0 < x_1 < \ldots < x_n \equiv b$ i formiramo sume 
	$$\begin{aligned}
		\underline{S}(f,d,a,b) = \sum^{n-1}_{i=0} m_i(x_{i+q}-x_i)
	\end{aligned}
	\quad \text{ i } \quad
	\begin{aligned}
		\overline{S}(f,d,a,b) = \sum^{n-1}_{i=0} M_i(x_{i+q}-x_i)
	\end{aligned}$$
	gde $m_i$ i $M_i$ označavaju infinum i supremum funkcije $f$ na segmentu $[x_i,x_{i+1}]; \; i=\overline{0, n-1}$, $\underline{S}$ i $\overline{S}$ se nazivaju \textbf{donja} i \textbf{gornja Darbuova suma} funkcije $f$ na segmentu $[a,b]$.\\
	\textbf{Napomena:} Ako je $f$ neprekidna funkcija na $[a,b]$ onda $f$ na segmentima $[x_i,x_{i+1}]$ (u nekim tačkama $\xi_i$) dostiže svoj infinum $m_i$ i supremum $M_i$, pa su sume $\underline{S}$ i $\overline{S}$ specijalni slučajevi opšteg pojma integralne sume.
\end{definition}
\begin{theorem}
	Ograničena funkcija $f$ je integrabilna na $[a,b]$ ako i samo ako je: $$\lim_{||d||\to 0} \left(\overline{S}(f,d,a,b) - \underline{S}(f,d,a,b)\right) = 0$$
\end{theorem}
\begin{theorem}
	Ako je funkcija $f$ \textbf{integrabilna} na segmentu $[a,b]$ tada je $f$ \textbf{ograničena} na $[a,b]$ \\
	\textbf{Napomena:} obrnuto ne važi
\end{theorem}
\begin{theorem}
	Ako je funkcija $f$ \textbf{neprekidna} na segmentu $[a,b]$ tada je \textbf{integrabilna} na $[a,b]$
\end{theorem}
\begin{theorem}
	Ako je funkcija $f$ \textbf{definisana} i  \textbf{ograničena} na segmentu $[a,b]$ i ako na $[a,b]$ ima \textbf{konačno mnogo} tačaka prekida, tada je $f$ \textbf{integrabilna} na $[a,b]$
\end{theorem}
\begin{theorem}
	Ako je funkcija $f$ \textbf{monotona} na segmentu $[a,b]$ tada je $f$ \textbf{integrabilna} na $[a,b]$.
\end{theorem}


\subsection{Svojstva određenog (Rimanovog) integrala}
\begin{enumerate}[label=\textbf{\arabic*.)}]
	\item
		\begin{theorem}
			Ako je funkcija $f$ integrabilna na segmentu $[a,b]$ gde je $a<b \land a,b \in \mathbb{R}$ onda $$\exists \int \limits^b_a f(x) \, \mathrm{d}x = - \int \limits^a_b f(x) \, \mathrm{d}x$$
		\end{theorem}
	\item 
		\textbf{Linearnost integrala}
		
		\begin{theorem}
				Neka su funkcije $f$ i $g$ integrabilne na segmentu $[a,b]$ i neka je funkcija $h(x) = \alpha f(x)+ \beta g(x)$ gde su $\alpha, \beta \in \mathbb{R}$ integrabilna na $[a,b]$
				$$\int\limits^b_a \left(\alpha f(x) + \beta g(x)\right) \, \mathrm{d} x = \alpha \int \limits^b_a f(x)\, \mathrm{d} x + \beta \int \limits^b_a g(x) \, \mathrm{d} x$$
		\end{theorem}
		\begin{proof}
			Funkcija $f$ je integrabilna $\implies \int \limits^b_a f(x) \, \mathrm{d}x = \lim_{||d|| \to 0} \sum (x_{i+1}-x_i)f(\xi_i)$\\
			Funkcija $g$ je integrabilna $\implies \int \limits^b_a g(x) \, \mathrm{d}x = \lim_{||d|| \to 0} \sum (x_{i+1}-x_i)g(\xi_i)$\\
			Funkcija $h$ je integrabilna $\implies \int \limits^b_a h(x) \, \mathrm{d}x =$\\
			\begin{gather*}
				=\lim_{||d|| \to 0} \sum (x_{i+1}-x_i)(\alpha f(\xi_i) + \beta g(\xi_i)) =\\
				= \lim_{||d|| \to 0} \left(\sum (x_{i+1}-x_i)\alpha f(\xi_i) + \sum (x_{i+1}-x_i)\beta g(\xi_i)\right) =\\
				= \lim_{||d|| \to 0} \sum (x_{i+1}-x_i)\alpha f(\xi_i) + lim_{||d|| \to 0}\sum (x_{i+1}-x_i)\beta g(\xi_i) =\\
				= \alpha\lim_{||d|| \to 0} \sum (x_{i+1}-x_i) f(\xi_i) + \beta \lim_{||d|| \to 0}\sum (x_{i+1}-x_i) g(\xi_i) =\\
				= \alpha \int \limits^b_a f(x) \, \mathrm{d}x + \beta \int 	\limits^b_a g(x) \, \mathrm{d}x
			\end{gather*}
		\end{proof}
	\item
		\textbf{Aditivnost integrala}
		\begin{theorem}
			Za bilo koje $c\in [a,b]$ važi $$\int\limits^b_a f(x) \, \mathrm{d} x = \int\limits^c_a f(x) \, \mathrm{d} x + \int\limits^b_c f(x) \, \mathrm{d} x$$
		\end{theorem}
	\item
		\textbf{Modularna nejednakost}
		\begin{theorem}
			Funkcija $|f(x)|$ je integrabilna na $[a,b]$ i važi:
			$$\left| \int \limits^b_a f(x) \, \mathrm{d} x \right| \leq \int \limits^b_a |f(x)| \, \mathrm{d} x$$ 
		\end{theorem}
	\item \begin{theorem}
			Funkcija $f(x)g(x)$ je integrabilna na $[a,b]$.
		\end{theorem}
	\item \begin{theorem}
			Funkcija $f(x)$ je integrabilna na $[\alpha, \beta] \subseteq [a,b]$
		\end{theorem}
	\item
		\begin{theorem}
			Ako je funkcija f integrabilna i $(\forall x \in [a,b]) f(x)=f_1(x)$ osim u konačno mnogo tačaka $c_1, c_2, \ldots, c_n\in[a,b]$ gde $f(c_i) \centernot = f_1(c_i)\; i= \overline{1, n}$ tada je funkcija $f_1(x)$ integrabilna na $[a,b]$ i važi:
			$$\int \limits^b_a f(x) \, \mathrm{d} x = \int \limits^b_a f_1(x) \, \mathrm{d} x$$
		\end{theorem}
		\begin{proof}
			Pretpostavimo da se $f$ i $f_1$ razlikuju u jednoj tački $c \in [a,b]$
			$$(\forall x \in [a,b])x\centernot = c \implies f(x)=f_1(x)$$
			Izaberimo proizvoljnu podelu $d$ segmenta $[a,b]$
			$$d=(x_0, x_1, \ldots,x_n, \xi_1, \xi_2, \ldots, \xi_n)$$
			Posmatrajmo integralne sume $S(f, d, a, b)$ i $S(f_1, d, a, b)$
			Ako tačka $c$ nije tačka podele $d$, tj. $(\forall \xi \in d) \xi\centernot = c$ onda su obe sume jednake. 
			Sume se razlikuju za $f$ i $f_1$ samo u slučaju da je $(\exists \xi \in d) c = \xi$ i tada bi se razlika ogčedača u sledećem sabirku:
			$$f(c)(x_i-x_{i-1}) \text{ i } f_1(c)(x_i-x_{i-1})$$
			Oba sabirka međutim teže nuli kada norma podele teži nuli $||d||\to 0$
			$$\implies \lim_{||d||\to 0} S(f, d, a, b) = \int \limits^b_a f(x) \, \mathrm{d}dx = \lim_{||d||\to 0} S(f_1, d, a, b) = \int \limits^b_a f_1(x) \, \mathrm{d}dx  $$
			Ako imamo konačan broj tačaka za koje se $f$ i $f_1$ razlikuju na $[a,b]$ tada za svaku od njih ponovimo isti postupak kao za tačku $c$
			$$\implies \int \limits^b_a f(x) \, \mathrm{d}x = \int \limits^b_a f_1(x) \, \mathrm{d}x$$
		\end{proof}
	\item
		\begin{theorem}
			\begin{enumerate}[label = \arabic*)]
				\item
					$(\forall x \in [a,b])f(x)\geq 0 \implies \int \limits^b_a f(x) \, \mathrm{d}x \geq 0$
				\item
					$(\forall x \in [a,b])f(x)> 0 \implies \int \limits^b_a f(x) \, \mathrm{d}x > 0$	
			\end{enumerate}
		\end{theorem}
	\item
		\textbf{Monotonost integrala}
		\begin{enumerate}[label = \arabic*)]
				\item
					$(\forall x \in [a,b])f(x)\leq g(x) \implies \int \limits^b_a f(x) \, \mathrm{d}x \leq \int \limits^b_a g(x) \, \mathrm{d}x$
				\item
					$(\forall x \in [a,b])f(x)< g(x) \implies \int \limits^b_a f(x) \, \mathrm{d}x < \int \limits^b_a g(x) \, \mathrm{d}x$	
		\end{enumerate}
		\item
			\begin{theorem}
				\textbf{(Stav o srednjoj vrednosti)}\\
				Neka je funkcija $f$ integrabilna na segmentu $[a,b]$ i neka $(\forall x \in [a,b]) m \leq f(x) \leq M$. Tada:
				$$(\exists \mu \in [m,M]) \int \limits^b_a f(x) \, \mathrm{d}x = \mu (b-a)$$
			\end{theorem}
			\begin{proof}
				\begin{gather*}
					(\forall x \in [a,b]) m \leq f(x) \leq M\\
					\int \limits^b_a m \, \mathrm{d}x \leq \int \limits^b_a f(x) \, \mathrm{d}x \leq \int \limits^b_a M \, \mathrm{d}x \\
					m(b-a)\leq\int \limits^b_a f(x) \, \mathrm{d}x\leq 	M(b-a)\\
					\int \limits^b_a f(x) \, \mathrm{d}x = \mu (b-a) \text{ gde je } m\leq\mu \leq M
				\end{gather*}
			\end{proof}
			\begin{corollary}
				Ako je $f$ neprekidna na $[a,b]$ tada:
				$$(\exists c \in [a,b])f(c) = \mu \implies \int \limits^b_a f(x) \, \mathrm{d}x = f(c)(b-a)$$
				Ako je funkcija integrabilna na $[a,b]$ srednja vrednost funkcije na $[a,b]$ je 
				$$f_s = \frac{1}{b-a}\int \limits^b_a f(x) \, \mathrm{d}x$$
			\end{corollary}
			\begin{proof}
				Iz diferencijalnog računa: $f$ je neprekidna na $[a,b]$
				$$\implies c \in [a,b] f(c) = \mu$$
			\end{proof}
\end{enumerate}


\subsection{Geometrijske primene određenog integrala}


\subsubsection{Površine ravnih likova}
\begin{theorem}
	Površina $P$ koja je ograničena neprekidnom funkcijom $f(x)\geq 0$ i odsečcima pravih $x=a$, $x=b$ i $y=0$ određena je formulom:
	$$P  = \int \limits^b_a f(x) \, \mathrm{d}x$$
\end{theorem}

\subsubsection{Dužina luka krive}
\begin{theorem}
	Neka je u ravni data kriva $(a\leq x \leq b) y=f(x)$ i neka su $f(x)$ i $f'(x)$ neprekidne na $[a,b]$. Dužina luka krive nad segmentom $[a,b]$ je:
	$$L = \int \limits^b_a \sqrt{1+(f'(x))^2} \, \mathrm{d}x = \int \limits^b_a \sqrt{1+(\frac{\mathrm{d}y}{\mathrm{d}x})^2} \, \mathrm{d}x = \int \limits^b_a \sqrt{\mathrm{d}x^2 + \mathrm{d}y^2} \, \mathrm{d}x$$
\end{theorem}

\subsubsection{Površina obrtnoh tela}
\begin{theorem}
	Neka su funkcije $f$ i $f'$ neprekidne na odsečku $[a,b]$. Ako se krivoliniski trapez, čije su stranice segment $[a,b]$, delovi pravih $x=a$, $x=b$ i kriva $(a\leq x \leq b) y=f(x)$ obrće oko $x$-ose, dobija se obrtno telo. Površina tog tela je:
	$$2\pi \int \limits^b_a f(x) \sqrt{1+(f'(x))^2}\, \mathrm{d}x$$
\end{theorem}


\subsection{Veza između određenog i neodređenog integrala}
\begin{theorem}
	Neka je funkcija $f$ integrabilna na $[a,b]$ i neka je $x\in [a,b]$ i, prema tome, neka je funkcija $\varphi = \int \limits^x_a f(t) \, \mathrm{d}t$. Pod ovim uslovima važi:
	\begin{enumerate}[label = \arabic*)]
		\item
			Funkcija $\varphi(x)$ je neprekidna na $[a,b]$.
		\item
			Ako je $f$ neprekidna na $[a,b]$ tada $(\forall x \in [a,b]) \varphi'(x) = f(x)$ ($\varphi$ je jedna primitivna funkcija funkcije $f$ na $[a,b]$)
	\end{enumerate}
\end{theorem}
\begin{proof}
	\begin{enumerate}[label = \arabic*)]
		\item
			Neka je $x\in [a,b]$ i neka je $\Delta x$ takvo da je $x+\Delta x \in [a,b]$.\\
			Ako je $\Delta x > 0$ (analogno se pokazuje za $\Delta x <0$, tada $[x+\Delta x] \subset [a,b]$) posmatrajmo segment $[x,x+\Delta x]\subset [a,b]$
			$$\varphi(x+\Delta x) - \varphi(x) = \int \limits^{x+\Delta x}_a f(t) \, \mathrm{d}t - \int \limits^x_a f(t) \, \mathrm{d}t=$$
			$$= \int \limits^{x+\Delta x}_a f(t) \, \mathrm{d}t + \int \limits^a_x f(t) \, \mathrm{d}t=\int \limits^{x+\Delta x}_x f(t) \, \mathrm{d}t$$
			Na osnovu teorema sa kojima smo se vež upoznali važi sledeći niz implikacija:\\
			$f$ je integrabilna na $[a,b] \implies f$ je integrabilna na $[x,x+\Delta x]\subset[a,b] \implies f$ je ograničena na $[x,x+\Delta x] \implies (\exists m_1, M_1)(\forall t \in [x, x+\Delta x]) m_1\leq f(t) \leq M_1$ Na osnovu teoreme o srednjoj vrednosti za određeni integral sledi: $(\exists \mu) m_1\leq \mu \leq M_1$ takvo da je 
			\begin{gather*}
				\int \limits^{x+\Delta x}_x f(t) \, \mathrm{d}t = \mu \Delta x\\
				\varphi(x+\Delta x)- \varphi(x) = \mu \Delta x\\
				 \lim_{\Delta x \to 0} (\varphi(x+\Delta x)-\varphi(x)) = \lim_{\Delta x \to 0} \mu \Delta x = \mu \lim_{\Delta x \to 0}\Delta x = 0\\
				\lim_{\Delta x \to 0} (\varphi(x+\Delta x)-\varphi(x)) = 0\\
				\implies F \text{ je neprekidna funkcija } \forall x \in [a,b]
			\end{gather*}		
		\item 
			$f$ je neprekidna na $[a,b] \implies f$ je neprekidna i na $[x, x+\Delta x] \in [a,b]$
			\begin{gather*}
				\implies ( \exists c \in [x, x+ \Delta x]) f(c) = \mu \\
				c = x+\theta \Delta x; \; \theta \in [0,1]\\
				\varphi(x+\Delta x)-\varphi(x) = \mu \Delta x\\
				\frac{\varphi(x+\Delta x)-\varphi(x)}{\Delta x} = f(c) = f(x+ \theta \Delta x)\\
				\lim_{\Delta x \to 0}	\frac{\varphi(x+\Delta x)-\varphi(x)}{\Delta x} = \lim_{\Delta x \to 0} f(x+ \theta \Delta x)\\
				\varphi'(x) = f(x); \; \forall x \in [a,b]\\
			\end{gather*}
			\begin{corollary}
				$\varphi(x)$ je jedna primitivna funkcija funkcije $f$
				$$(\exists c \in \mathbb{R}) \varphi (x) = F(x) +c$$
				Za $x=a\; : \;\varphi(a) = F(a)+c = \int \limits^a_a f(x) \, \mathrm{d} x = 0 \implies c = -F(a)$\\
				Za $x=b\; : \;\varphi(b) = F(b)+c = \int \limits^b_a f(x)\, \mathrm{d}x$
				$$\underbrace{\int \limits^b_a f(x) \, \mathrm{d} x = F(b)-F(a) = F(x)|^b_a}_{\text{Njutn-Lajbnicova formula}}$$
			\end{corollary}
	\end{enumerate}
\end{proof}
\begin{theorem}
	Ako je funkcija $f$ neprekidna tada važi:
	$$\int \limits^b_a f(x) \, \mathrm{d} x = F(x)|^b_a = F(b)-F(a)$$ gde je $F$ bilo koja primitivna funkcija funkcije $f$.
\end{theorem}

\subsection{Parcijalna integracija i smena promenljive kod određenog integrala}
\begin{theorem}
	\textbf{(Parcijalna integracija kod određenog integrala)}\\
	Ako su funkcije $u(x)$, $v(x)$, $u'(x)$ i $v'(x)$ neprekidne na $[a,b]$ tada je:
	$$\int\limits^b_a u(x)\, \mathrm{d} v(x) = u(x)v(x)|^b_a - \int\limits^b_a v(x)\, \mathrm{d} u(x) = u(b)v(b) - u(a)v(a) - \int\limits^b_a v(x)\, \mathrm{d} u(x)$$
\end{theorem}
\begin{theorem}
	\textbf{(Smena promenljivo kod određenog integrala)}\\
	Ako su ispunjeni sledeći uslovi:
	\begin{enumerate}[label = \alph*)]
		\item 
			funkcija $f(x)$ je neprekidna na segmentu $[a,b]$
		\item 
			funkcije $x=\varphi(t)$ i $\varphi'(t)$ su neprekidne na segmentu $[\alpha, \beta]$ gde je $a = \varphi(\alpha)$ i $b = \varphi(\beta)$ tada važi:
			$$\int \limits^b_a f(x) \, \mathrm{d}x = \int \limits^\beta_\alpha f(\varphi(t))\varphi'(t)\, \mathrm{d}t$$
	\end{enumerate}
	\textbf{Napomena:} U navedenoj teoremi formula smene promenljive izvedena je pod pretpostavkom da je funkcija $f$ neprekidna, što je i najčešći slučaj u primenama. U slučaju da je funkcija $\varphi$ strogo monotona formula važi i pod slabijom pretpostavkom da je $f$ integrabilna na $[a,b]$.
\end{theorem}
\begin{theorem}
	Ako je $f:\mathbb{R}\to \mathbb{R}$ neprekidna i periodična funkcija sa periodom $T$, tada važi:
	$$\int \limits^{a+T}_a f(x) \, \mathrm{d}x = \int \limits^T_0 f(x) \, \mathrm{d}x$$
\end{theorem}
\begin{proof}
$$\ldots \quad \text{?}$$
\end{proof}
\begin{theorem}
	Ako je funkcija $f$ neprekidna na $[-a, a]$ tada važi:
	$$\int \limits^a_{-a}f(x)\, \mathrm{d} x =
	\begin{cases}
		0 &\text{ako je }f\text{ neparna funkcija}\\
		2\int\limits^a_0 f(x)\, \mathrm{d} x \quad &\text{ako je }f\text{ parna funkcija}
	\end{cases}
	$$
\end{theorem}
\begin{proof}
$$\ldots \quad \text{?}$$
\end{proof}

\subsection{Nesvojsveni integral}
Postoje dva razloga zbog kojih integral nije Rimanov:
\begin{enumerate}[label = \arabic*.)]
	\item
		Oblast integracije je $\pm \infty$
	\item
		Integral ima singularne tačke u intervalu
\end{enumerate}
\begin{definition}
	\begin{enumerate}[label = \arabic*)]
		\item
			Neka je $f$ definisana na intervalu $[a, +\infty)$, i integrabilna na svakom konačnom $[\alpha, \beta]\subset[a,+\infty)$. Tada je $\int \limits^{+\infty}_af(x)\, \mathrm{d}x$ integral koji zovemo nesvojstvenim (nepravim) i definiše se kao:
			$$\int \limits^{+\infty}_a f(x)\, \mathrm{d}x \triangleq \lim_{\beta\to + \infty}\int\limits^\beta_a f(x)\, \mathrm{d}x$$
		\item	
			$\int \limits^b_{-\infty} f(x)\, \mathrm{d}x \triangleq \lim_{\alpha \to -\infty} \int \limits^b_\alpha f(x)\, \mathrm{d}x$
		\item
			$\int \limits^{+\infty}_{-\infty} f(x)\, \mathrm{d}x = \int \limits^c_{-\infty} f(x)\, \mathrm{d}x + \int \limits^{+\infty}_c f(x)\, \mathrm{d}x \triangleq \lim_{\alpha \to -\infty} \int \limits^c_\alpha f(x)\, \mathrm{d}x + \lim_{\beta \to +\infty} \int \limits^\beta_c f(x)\, \mathrm{d}x$
	\end{enumerate}
\end{definition}
\begin{definition}
	\begin{enumerate}[label = \alph*)]
		\item
			Neka je $f$ definisana na segmentu $[a,b)$ i integrabilna na svakom konačnom $[\alpha, \beta]\subset[a,b)$ tada važi:
			$$\underbrace{\int\limits^b_a f(x)\, \mathrm{d}x}_{\text{Nesvojstven}} = \underbrace{\lim_{\beta \to b_-} \int \limits^\beta_a f(x)\, \mathrm{d}x}_{\text{Konačan, beskonačan ne postoji}}$$
		\item
			Neka je $f$ definisana na segmentu $(a,b]$ i integrabilna na svakom konačnom $[\alpha, \beta]\subset(a,b]$ tada važi:
			$$\underbrace{\int\limits^b_a f(x)\, \mathrm{d}x}_{\text{Nesvojstven}} = \underbrace{\lim_{\alpha \to a_+} \int \limits^b_\alpha f(x)\, \mathrm{d}x}_{\text{Konačan, beskonačan ne postoji}}$$
	\end{enumerate}
\end{definition}

	\section{Diferencijalne jednačine}
- Principijalno ne znamo da ih rešavamo\\
- Neke i možemo da rešimo (obratiti pažnju na klasifikaciju)
\subsection{Obične diferencijalne jednačine}
\begin{definition}
	Implicitan izraz $f(x,y,y',\ldots,y^{(n)})=0$ gde je $y:(a,b)\to \mathbb{R}$ nepoznata, $n$ puta diferencijabilna funkcija nezavisno promenljive $x$, naziva se \textbf{obična diferencijabilna jednačina} reda $n$ ako u njoj efektivno učestvuje izvod $y^{(n)}$.\\
	\textbf{Napomena:} nije obavezno pojavljivanje svih članova, ali je obavezno pojavljivanje $n$-tog izvoda.
\end{definition}
\begin{definition}
	\textbf{Rešenje} diferencijalne jednačine na intervalu $I$ je svaka funkcija $y$ definisana na $I$ koja jednačinu svodi na identitet.\\
\end{definition}
\begin{definition}
	\textbf{Opšte rešenje} diferencijalne jednačine $n$-tog reda je svaka funkcija $y$ definisana sa $G(x,y,c_1,c_2,\ldots,c_n)$ gde su $c_1,c_2,\ldots,c_n$ proizvoljne konstante iz $\overline{\mathbb{R}}$, tako da:
	\begin{enumerate}[label=\arabic*)]
		\item
			$y$ jeste rešenje jednačine
		\item
			Polazna jednačine se može dobiti iz izraza $G$
	\end{enumerate}
	\textbf{Napomena:} Opšte rešenje ne sadrži sva rešenja diferencijalne jednačine
\end{definition}
\begin{definition}
	\textbf{Partikularno rešenje} diferencijalne jednačine $n$-tog reda je svako njeno rešenje koje je obuhvaćeno opštim rešenjem, tj. koje se može dobiti iz opšteg rešenja za neke konkretne vrednosti konstanti.
\end{definition}
\begin{definition}
	\textbf{Singularno rešenje} diferencijalne jednačine $n$-tog reda je svako rešenje koje nije obuhvaćeno opštim rešenjem.
\end{definition}
\begin{definition}
	Partikularno rešenje diferencijalne jednačine $n$-tog reda određeno uslovima:
	$$\left.
	\begin{aligned}
		y(x_0) &= y_0\\
		y'(x_0) &= y_1\\
		\vdots\\
		y^{(n-1)}(x_0) &= y_{n-1}\\
		y^{(n)}(x_0) &= y_n
	\end{aligned}
	\right\} \text{Košijevi (početni) uslovi}
	$$
	naziva se \textbf{Košijevo rešenje} diferencijalne jednačine za početne uslove.\\
	\textbf{Napomena:} Ovaj problem drugačije se naziva Košijev problem ili problem početnih uslova. Rešava se $n$ jednačina ($n$ uslova) sa $n$ nepoznatih ($n$ konstanti). Ukoliko je problem korektno zadat onda ima jedinstveno rešenje.
\end{definition}


\subsection{Diferencijalne jednačine prvog reda}


\subsubsection{Diferencijalne jednačina kod kod kojih promenljive mogu da se razdvoje}
\textbf{Opšti izraz:}
\begin{gather*}
	f(x)\, \mathrm{d}x +g(y)\, \mathrm{d}y = 0\\
	y' = \varphi(x)\xi(y)
\end{gather*}
gde su funkcije $f$, $g$, $\varphi$ i $\xi$ definisane i neprekidne na nekom intervalu $I$ na kom se traži rešenje ove jednačine.
\textbf{Opšte rešenje:}
$$\int f(x) \, \mathrm{d}x + \int g(y) \, \mathrm{d}y = c$$
\textbf{Partikularno rešenje Košijevog problema:}
$$\int \limits^x_{x_0} f(x) \, \mathrm{d}x +\int \limits^y_{y_0} g(y) \, \mathrm{d}y = 0$$
\textbf{Napomena:} Na kraju rešavanja broj konstanti mora biti jednak redu diferencijalne jednačine.


\subsubsection{Diferencijalne jednačine oblika $y'=g(ax+by)$}
\textbf{Opšti izraz:}
$$y'=g(ax+by)$$
gde su funkcije $y$ i $g$ definisane i neprekidne na nekom intervalu $I$ na kom se traži rešenje ove jednačine.
\textbf{Opšte rešenje:}
\begin{enumerate}[label = \alph*)]
	\item
		$b=0 \implies$ ovo je jednačina kod koje promenljive mogu da se razdvoje
	\item
		$b \centernot = 0 \implies$ u ovoj jednačini se uvodi smena zavisno promenljive $y$:
		\begin{align*}
			\begin{aligned}[c]
				ax+by=z\\
				by = z-ax\\
				y = \frac{z-ax}{b}\\
				y' = \frac{z'-a}{b}\\	
				y' = g(z)\\	
			\end{aligned}
			\quad \quad
			\begin{aligned}[c]
				\frac{z'-a}{b} = g(z)\\
				z' = bg(z)+a\\
				\frac{\mathrm{d}z}{\mathrm{d}x} = bg(z)+a\\
				\mathrm{d}x = \frac{\mathrm{d}z}{bg(z)+a}\\
				\int\frac{\mathrm{d}z}{bg(z)+a} = \int \mathrm{d}x + c\\
			\end{aligned}\\		
		\end{align*}
		$$\underbrace{\int\frac{\mathrm{d}z}{bg(z)+a} = x + c}_{\text{Rešavanjem dobija se opšte rešenje}}$$
\end{enumerate}

\subsubsection{Homogena diferencijalna jednačina prvog reda}
\textbf{Opšti izraz:}
$$y' = f\left(\frac{y}{x}\right)$$
gde je funkcija $f$ definisana na intervalu $I$ na kom se traži rešenje ove jednačine. I $f(u)\centernot \equiv u \implies$ nije identičko preslikavanje\\
\textbf{Opšte rešenje:}
\begin{align*}
	\begin{aligned}[c]
		z = \frac{y}{x}\\
		y = xz\\
		y' = z+z'x\\
		y' = f\left(\frac{y}{x}\right)\\
	\end{aligned}
	\quad \quad
	\begin{aligned}[c]
		z'x+z = f(z)\\
		\frac{\mathrm{d}z}{\mathrm{d}x}x = f(z)-z\\
		\frac{\mathrm{d}z}{f(z)-z} = \frac{\mathrm{d}x}{x}\\
		\int \frac{\mathrm{d}z}{f(z)-z} = \int \frac{\mathrm{d}x}{x}+c\\
	\end{aligned}\\	
\end{align*}
$$\underbrace{\int \frac{\mathrm{d}z}{f(z)-z} = \ln|x|+c}_{\text{Rešavanjem dobija se opšte rešenje}}$$


\subsubsection{Diferencijalna jednačina prvog reda oblika $y' = f\left(\frac{a_1x+b_1y+c_1}{a_2x+b_2y+c_2}\right)$}
\textbf{Opšti izraz:}
$$y' = f\left(\frac{a_1x+b_1y+c_1}{a_2x+b_2y+c_2}\right)$$
gde je funkcija $f$ definisana i neprekidna na nekom intervalu $I$ na kom se traži rešenje ove jednačine
\textbf{Opšte rešenje:}
\begin{enumerate}[label = \alph*)]
	\item
		$a_1 = a_2 = b_1 = b_2 = 0$ - Nije poželjno da istovremeno budu jednake nuli.
	\item
		$a_1=a_2=0 \lor b_1=b_2=0 \lor a_1=b_2=0 \lor b_1=a_2=0 \implies$ Diferencijalna jednačina kod koje promenljive mogu da se razdvoje.
	\item
		$c_1=c_2=0 \implies$ Homogena diferencijalna jednačina.
	\item
		$\left|
		\begin{matrix}
			a_1 & b_1\\
			a_2 & b_2
		\end{matrix}
		\right| = 0 \implies \frac{a_1}{a_2}=\frac{b_1}{b_2}\implies$ Koeficijenti $ a_1$, $a_2$, $b_1$ i $b_2$ su proporcionalni $\implies$ uvođenjem smene $a_1x+b_1y=z$ svodi se na diferencijalnu jednačinu oblika $y'=g(a_1x+b_1y)$ koju znamo da rešimo.
	\item
		$\left|
		\begin{matrix}
			a_1 & b_1\\
			a_2 & b_2
		\end{matrix}
		\right| \centernot = 0 \implies \frac{a_1}{a_2}\centernot=\frac{b_1}{b_2}\implies$Koeficijenti $ a_1$, $a_2$, $b_1$ i $b_2$ nisu proporcionalni $\implies$ smetaju $c_1$ i $c_2$, pa se uvode dve smene:
		\begin{enumerate}[label = \arabic*)]
			\item
				smena nezavisno promenljive $x=u+\alpha$
			\item
				smena zavisno promenljive $y=v+\beta$		
		\end{enumerate}				
		gde su $\alpha$ i $\beta$ konstante, rešenja sistema linearnih jednačina:
		$$\left.
		\begin{aligned}
			a_1\alpha+b_1\beta +c_1=0\\
			a_2\alpha+b_2\beta +c_2=0
		\end{aligned}
		\right\}a_i,\, b_i,\, c_i \text{ su koeficijenti iz polazne jednačine}$$
		$\det \centernot = 0\implies$ sistem ima jedinstveno rešenje, $\alpha$ i $\beta$ daju nove promenljive i jednačina se svodi na homogenu diferencijalnu jednačinu prvog reda.
\end{enumerate}

\subsubsection{Linearna diferencijalna jednačina prvog reda}
\textbf{Opšti izraz:}
$$y'+P(x)y=Q(x)$$
gde su $P$ i $Q$ funkcije definisane i neprekidne na nekom intervalu $I$ na kom se traži rešenje ove jednačine
\textbf{Opšte rešenje:}
\begin{enumerate}[label = \alph*)]
	\item 
		Homogena\\
		$Q(x) = 0 \implies$ promenljive mogu da se razdvoje
		\begin{align*}
			\begin{aligned}[c]
				-P(x)y=y'\\
				\frac{\mathrm{d}y}{y} = -P(x)\,\mathrm{d}x\\
			\end{aligned}
			\quad \quad
			\begin{aligned}[c]
				\int \frac{\mathrm{d}y}{y} = - \int P(x)\,\mathrm{d}x+c\\
				\ln|y| = -\int P(x)\mathrm{d}x+\ln c\\	
			\end{aligned}
		\end{align*}
		$$y = ce^{-\int P(x) \mathrm{d}x}$$
	\item
		Nehomogena\\
		\begin{align*}
			\begin{aligned}[c]
				y'+P(x)y=Q(x)\\
				y'e^{\int P(x)\mathrm{d}x} + yP(x)e^{\int P(x)\mathrm{d}x} = Q(x)e^{\int P(x)\mathrm{d}x}\\
				\left(ye^{\int P(x)\mathrm{d}x}\right)'=Q(x)e^{\int P(x)\mathrm{d}x}
			\end{aligned}
			\quad \quad
			\begin{aligned}[c]
				\int\left(ye^{\int P(x)\mathrm{d}x}\right)'=\int Q(x)e^{\int P(x)\mathrm{d}x}\\
				ye^{\int P(x)\mathrm{d}x} = \int Q(x)e^{\int P(x)\mathrm{d}x}\,\mathrm{d}x +c\\
			\end{aligned}
		\end{align*}
		$$y = e^{-\int P(x)\mathrm{d}x}\left(c+ \int Q(x)e^{\int P(x)\mathrm{d}x} \,\mathrm{d}x \right)$$
		Košijev problem, opšte rešenje za linearnu nehomogenu diferencijalnu jednačinu prvog reda:
		$$y = e^{-\int\limits^x_{x_0} P(x)\mathrm{d}x}\left(y_0+ \int\limits^x_{x_0} Q(x)e^{\int\limits^x_{x_0} P(x)\mathrm{d}x} \, \mathrm{d}x \right)$$
\end{enumerate}

\subsubsection{Bernulijeva diferencijalna jednačina}
\textbf{Opšti izraz:}
$$y'+P(x)y=Q(x)y^m$$
gde je $m \in \mathbb{R}\setminus \{0,1\}$, funkcije $P$ i $Q$ definisane i neprekidne na nekom intervalu $I$ na kom se rešava jednačina.\\
\textbf{Opšte rešenje:}
\begin{align*}
	\begin{aligned}[c]
		z = y^{1-m}\\
		z' = (1-m)y^{-m}y'\\
		y'+P(x)y=Q(x)y^m
	\end{aligned}
	\quad \quad
	\begin{aligned}[c]
		\frac{y'}{y^m}+P(x)y^{1-m}=Q(x)\\
		(1-m)y^{-m}y' + P(x)y^{1-m}=(1-m)Q(x)
	\end{aligned}
\end{align*}
$$z'+z\underbrace{P(x)}_{P_1(x)}=\underbrace{(1-m)Q(x)}_{Q_1(x)}$$ - linearna diferencijalna jednačina prvog reda

\subsubsection{Rikatijeva jednačina}
\textbf{Opšti izraz:}
$$y' = P(x)y^2+Q(x)y+R(x)$$
gde su funkcije $P$, $Q$ i $R$ definisane i neprekidne na nekom intervalu $I$ na kom se rešava jednačina.\\
\textbf{Opšte rešenje:}
Izaberimo takvu smenu da uklonimo $R(x)$ i dobijamo Bernulijevu diferencijalnu jednačinu za $m=2$\\
$y^2 \implies m= 2 \implies$ potrebno nam je jedno partikularno rešenje, od toga polazimo $(y_p)$\\
$y=y_p+\frac{1}{z} \implies$ smena zavisno promenljive\\
$y' = y'_p - \frac{z'}{z^2} \implies$ izvod nije nula\\
\begin{gather*}
	y'_p - \frac{z'}{z^2} = ((y_p)^2 + \frac{2y_p}{z} + \frac{1}{z^2}) P(x) + Q(x)(y_p + \frac{1}{z}) + R(x)\\
	y'_p - \frac{z'}{z^2} = y_p^2 P(x) + 2P(x)\frac{y_p}{z} + \frac{P(x)}{z^2} + Q(x)y_p + \frac{Q(x)}{z} + R(x)\\
	y'_p = P(x)y_p^2 + Q(x)y_p + R(x)\\
	z' = -2zP(x)y_p - P(x) - zQ(x) \implies\\
	z' + 2zP(x)y_pz+zQ(x)= -P(x)\\
	z' + z\underbrace{(2P(x)y_p+Q(x))}_{P_1} = \underbrace{-P(x)}_{Q_1}
\end{gather*} 
- linearna diferencijalna jednačina prvog reda

\subsection{Diferencijalne jednačine drugog reda}

\textbf{Opšti izraz:}
$$ F(x, y, y', y'') $$
gde je $F$ funkcija definisana i dva puta diferencijabilna na nekom intervalu $I$ na kom se jednačina rešava.

\subsubsection{Nepotpune diferencijalne jednačine drugog reda}
\begin{enumerate}[label = \textbf{\arabic*. slučaj}] 
	\item 
		$ F(y', y'') = 0 $ ili $ y'' = f(y') $\\
		- uvodi se smena zavisno promenljive $ p = y $\\
		\begin{align*}
			\begin{aligned}
				y' = p\\
				y'' = \frac{\dif p}{\dif x}\\
				y'' = f(p)\\
			\end{aligned}
			\quad \quad
			\begin{aligned}
			\frac{\dif p}{\dif x} = f(p)\\
			\int \frac{\dif p}{f(p)} = x + c \implies \varphi (x, p, c_1) = 0
			\end{aligned}
		\end{align*}
		Dolazi se do diferencijalne jednačine prvog reda $ \varphi (x, y, c_1) = 0 $, i na kraju rešavanjem te diferencijalne jednačine dolazi se do: $\psi (x, y, c_1, c_2) = 0$
	\item 
		$ F(x, y, y'') $\\
		- uvodi se smena zavisno promenljive $ p = y' $\\
		\begin{align*}
		\begin{aligned}
		y' = p\\
		y'' = \frac{\dif p}{\dif x}
		\end{aligned}
		\implies F(x, p, \frac{\dif p}{\dif x}) = 0
		\end{align*}
		Dolazi se do diferencijalne jednačine prvog reda $ \varphi (x, y', c_1) = 0 $, i na kraju rešavanjem te diferencijalne jednačine dolazi se do: $\psi (x, y, c_1, c_2) = 0$
	\item 
		$ F(y, y', y'') = 0 $\\
		- uvodi se smena zavisno promenljive $ y' = p $ (gubimo $ x $ tokom smene)\\
		\begin{align*}
		\begin{aligned}
		y' = p\\
		y'' = \frac{\dif p}{\dif x} \frac{\dif y}{\dif y} = y' \frac{\dif p}{\dif y} = p \frac{\dif p}{\dif y}\\
		\end{aligned}
		\implies F(y, p, p \frac{\dif p}{\dif x}) = 0
		\end{align*}
		Dolazi se do diferencijalne jednačine prvog reda $ \varphi (y, p, c_1) = 0 $, i na kraju rešavanjem te diferencijalne jednačine dolazi se do: $\psi (x, y, c_1, c_2) = 0$
\end{enumerate}

\subsection{Linearna diferencijalna jednačina $n$-tog reda}
\begin{definition}
	Diferencijalna jednačina oblika $y^{(n)} + f_1(x)y^{(n-1)} + \ldots + f_n(x)y = F(x)$ naziva se \textbf{linearna diferencijalna jednačina $n$-tog reda}. Pri tome podrazumevamo da su $f_i(x)$ i $F(x)$ definisane i neprekidne na posmatranom intervalu $I$.
	\begin{enumerate}[label = \arabic*.)]
		\item
			$F(x) \equiv 0$ - homogena jednačina
		\item 
			$F(x) \centernot \equiv 0$ - nehomogena jednačina
	\end{enumerate} 
	\Napomena ako su $f_i(x)$ gde je $i = \overline{1, n}$ konstante jednačina je sa konstantnim koeficijentima (u suprotnom je sa gunkcionalnim).
\end{definition}

\subsubsection{Homogena linearna diferencijalna jednačina}
	\begin{definition}
		$L^{n}(y) = y^{n} + d_1(x)y^{n-1} + \ldots + f_n(x)y = 0$ naziva se \textbf{linearni diferencijalni operator $n$-tog reda}
	\end{definition}
	Jedno rešenje (trivijalno) diferencijalne jednačine $L^{n}(y)$ je $y=0$, ali mi tražimo netrivijalna. Takođe se vidi: ako su $y_1(x)$ i $y_2(x)$ rešenja, tada je i $c_1y_1(x) + x_2 y_2(x)$ takođe rešenje date diferencijalne jednačine. Što se može uopštiti na proizvoljan broj rešenja. 
	\begin{gather}
		L^n (y_1) = 0 \land L^n(y_2) = 0\\
		L^n (y_1c_1 + y_2c_2) = c_1L^n(y_1) + x_2 L^n(y_2) = c_1\cdot 0 + c_2\cdot 0= 0+0 = 0
	\end{gather}
	\begin{definition}
		Funkcije $y_1(x), y_2(x), \ldots, y_n(x)$ definisane na intervalu $I$ mogu biti:
		\begin{enumerate}[label = \arabic*.)]
			\item 
				Linearno zavisne na $I$\\
				- ako postoje konstante $c_2, c_2, \ldots, c_n$ takve da $\sum_{k=0}^{n}c_k^2 \centernot 0 $ i da važi da je $\forall x \in I) c_1y_1(x) + c_2y_2(x) + \ldots + c_ny_n(x) = 0$
			\item 
				Linearno nezavisne na $I$\\
				- u suprotnom
		\end{enumerate}
	\Napomena u slučaju $n=2$ linearna nezavisnost se sodi na $\frac{y_1}{y_2} \centernot const$
	\end{definition}
	\begin{theorem}
		Ako su  $y_1(x), y_2(x), \ldots, y_n(x)$ rešenja jednačine $L^n(y) = 0$ i ako su funkcije  $y_1(x), y_2(x), \ldots, y_n(x)$ linearno nezavisne, tada je opšte rešenje date jednačine:
		$$ y =c_1y_1(x) + c_2y_2(x) + \ldots + c_ny_n(x) $$
		gde su $c1$ za $i = \overline{1, n}$ konstante
	\end{theorem}

\subsubsection{Homogena linearna diferencijalna jednačina drugog reda}
	\textbf{Opšti izraz: }
	$$ L^2(y) = 0 \; \lor \; y'' + P(x)y' + Q(x) y = 0 $$
	\begin{theorem}
		Liuvilova formula: Ako je $y_1(x)$ jedno netrivijalno partikularno rešenje linearne diferencijalne jednačine $y'' + P(x)y+ Q(x)y = 0$ tada je:
		$$y_2(x) = y_1(x) \int \frac{1}{y_1^2(x)} e^{-\int P(x) \dif x} \dif x$$
		takođe partikularno rešenje date jednačine linearno nezavisno od $y_1$.
	\end{theorem}
	\begin{proof}
		\begin{align*}
			\begin{gathered}
				y = y_1z\\
				y' = y'_1z + y_1 z'\\
				y'' = y''_1z + 2y'_1z' + z''y_1\\
				y''_1z+2y'_1z' + y_1z'' + P(x)y+_1z + P(x)y_1z' + Q(x)y_1z = 0\\
				(y''_1 + P(x)y'_1 + Q(x)y_1) + z'(2y'_1 + P(x)y_1) + z''y_1 = 0\\
				y''_1 + P(x)y'_1 + Q(x)y_1 = 0 \text{ partikularno, netrivijalno}\\
				z'' + z'(\frac{2y'_1}{y_1})+ P(x)) = 0\\
			\end{gathered}
			\quad \quad
			\begin{gathered}
				z' = x_1 e^{-\int (\frac{2y'_1}{y_1}+P(x)) \dif x} = \frac{\dif z}{\dif x}\\
				\frac{\dif z}{\dif x} = c_1 e^{-2\ln |y_1|-\int P(x) \dif x}\\
				\frac{\dif z}{\dif x} = c_1 e^{\ln \frac{1}{y_1^2}-\int P(x) \dif x}\\
				\frac{\dif z}{\dif x} = c_1 \frac{1}{y_1^2}e^{-\int P(x) \dif x}\\
				\dif z = \frac{c_1}{y_1^2} e^{-\int P(x) \dif x} \dif x\\
				z = \int \frac{c_1}{y_1^2} e^{-\int P(x) \dif x} \dif x + c_2\\
				y = y_1c_2 + c_1y_1 \int \frac{e^{- \int P(x) \dif x}}{y_1^2} \dif x\\
			\end{gathered}
		\end{align*}
		$$ y_2 = y_1 \int \frac{1}{y_1^2} e^{- \int P(x) \dif x} \dif x $$
	\end{proof}

\subsubsection{Nehomogena linearna diferencijalna jednačina drugog reda}
$$ L^n(y) = F(x) $$
U slučaju $n = 2$ snižavanje reda pomožu jednog partikularnog rešenja dovodi do linearne nehomogene jednačine prvog reda.
\begin{theorem}
	Neka je $y_h$ opšte ređenjee homogene jednačine $L^n(y) = 0$ i $y_p$ jedno partikularno rešenje nehomogene jednačine $L^n(y) = F(x)$. Tada je opšte rešenje nehomogene jednačine $L^n(y)=F(x)$ dato sa 
	$$y = y_h + y_p$$
\end{theorem}
\begin{proof}
	Neposredno se proverava da je $L^n(y_h + y_p) = L^n(y_h) + L^n(y_p) = 0 + F(x) = F(x)$ jer je $F(x) = L^n(y_p)$ i $0 = L^n(y_h)$
\end{proof}
\begin{theorem}
	\textbf{Metod varijacije konstanti (Lagranžov):} Neka je $y_h = c_1 y_1(x) + c_2 y_2(x) + \ldots + c_n y_n(x)$ opšte rešenje homogene jednačine $L^n(y) = 0$ tada je opšte rešenje nehomogene jednačine $L^n(y) = F(x)$ dato sa $y = c_1(x) y_1(x) + c_2(x) y_2(x) + \ldots + c_n (x) y_n(x)$ gde su $c_1(x), \, c_2(x), \, \ldots, \, c_n(x)$ funkcije čiji se izvodi nalaze rešavanjem sistema jednačina:
	\begin{gather*}
		c'_1(x) y_1(x) + c'_2(x) y_2(x) + \ldots + c'_n (x) y_n(x) = 0\\
		c'_1(x) y'_1(x) + c'_2(x) y'_2(x) + \ldots + c'_n (x) y'_n(x) = 0\\
		\vdots\\
		c'_1(x) y_1^{(n)}(x) + c'_2(x) y^{(n)}_2(x) + \ldots + c'_n (x) y^{(n)}_n(x) = F(x)\\
	\end{gather*}
	a zatim se same funkcije $c_1(x), \, c_2(x), \, \ldots, \, c_n(x)$ nalaze integracijom:
	\begin{gather*}
		\int c'_1(x) \dif = c_1(x) + c^*_1\\
		\int c'_2(x) \dif = c_2(x) + c^*_2\\
		\vdots\\
		\int c'_n(x) \dif = c_n(x) + c^*_n\\
	\end{gather*}
	Opšte rešenje se dobija kao $y = y_p + y_h$ gde je jedno partikularno rešenje  $y_p = y_1 c_1(x) + y_2 c_2 (x) + \ldots + y_n c_n (x)$.
\end{theorem}

\subsubsection{Linearna diferencijalna jednačina $n$-tog reda sa konstantnim koeficijentima}
\begin{definition}
	\underline{Homogena} jednačina $L^n(y) = y^{(n)} + p_1 y^{(n-1)} + \ldots + p_n y = 0$ gde su $p_i \in \overline{\realset}$ za $i = \overline{1, n}$ konstante, naziva se \textbf{\underline{homogena} linearna diferencijalna jednačina $n$-tog reda sa konstantnim koeficijentima}\\
\end{definition}
Ovaj tip diferencijalnih jednačina rešava se traženjem partikularnog rešenja u obliku $y = e^{\lambda x}$. Zamenom u jednačinu dobijamo $e^{\lambda x}(\lambda^n + p_1 \lambda^{n-1} + \ldots + p_n) = 0$
\begin{definition}
	Algebarska jednačina $\lambda^n + p_1 \lambda^{n-1} + \ldots + p_{n-1} \lambda + p_n = 0$ zove se karakteristična jednačina prethodno navedene diferencijalne jednačine.
\end{definition}
Karakteristična jednačina ima $n$ korena i svakom korenu odgovara po jedno partikularno rešenje, po sledećim pravilima:
\begin{enumerate}[label = \arabic*)]
	\item Svakom realnom, jednostavnom korenu $\lambda$ odgovara jedno partikularno rešenje $y_p = e^{\lambda x}$
	\item Svakom realnom korenu $\lambda$ kada $k > 1$ odgovara $k$ partikularnih rešenja $e^{\lambda x}, \, x e^{\lambda x}, \, \ldots, \, x^{k-1} e^{\lambda x}$
	\item Svakom paru kompleksnih, jednostavnih korena $\alpha \pm i \beta$ odgovaraju partikularna rešenja $e^{\alpha x}\cos \beta x, \, e^{\alpha x} \sin \beta x$.
	\item Svakom paru kompleksnih korena $\alpha \pm i \beta$ kada $k > 1$ odgovara $2k$ partikularnih rešenja:
	\begin{gather*}
		e^{\alpha x} \cos \beta x, \, xe^{\alpha x} \cos \beta x, \, \ldots, \, x^{k-1}e^{\alpha x} \cos \beta x, \, \\
		e^{\alpha x} \sin \beta x, \, xe^{\alpha x} \sin \beta x, \, \ldots, \, x^{k-1}e^{\alpha x} \sin \beta x, \, \\
	\end{gather*}
\end{enumerate}
Primenom ovih pravila na sve korene karakteristične jednačine dobija se skup od ukupno $n$ linearno nezavisnih rešenja $y_1, y_2, \ldots, y_n$ diferencijalne jednačine pa je opšte rešenje homogene diferencijalne jednačine $y_h = c_1 y_1(x) + c_2 y_2(x) + \ldots + c_n y_n(x)$
\begin{definition}
	\underline{Nehomogena} jednačina $L^n(y) = y^{(n)} + p_1 y^{(n-1)} + \ldots + p_n y = F(x)$ gde su $p_i \in \overline{\realset}$ za $i = \overline{1, n}$ konstante, naziva se \textbf{\underline{nehomogena} linearna diferencijalna jednačina $n$-tog reda sa konstantnim koeficijentima}
\end{definition}
Rešavanje nehomogene jednačine sastoji se iz 2 koraka:
\begin{enumerate}[label = \arabic*)]
	\item Reši se homogena jednačina (odredi se $y_h$)
	\item Određuje se opšte rešenje nehomogene jednačine metodom varijacije konstantni
\end{enumerate}
\Napomena ako je funkcija $F(x)$ specijalnog oblika (izvodi su istog oblika kao sama funkcija - polinomska/eksponencijalna/trigonometrijska funkcija, ili su zbirovi/proizvodi tih funkcija), partikularno rešenje $y_p$ može se odrediti metodom neodređenih koeficijenata (više o tome u 3.4.5) tada je opšte rešenje $y = y_h + y_p$
\subsubsection{Metod neodređenih koeficijenata}
	Metod neodređenih koeficijenata sastoji se u tome da se partikularno rešenje nehomogene jednačine pretpostavi u obliku koji je sličan obliku funkcije $F(x)$ ali sa neodređenim koeficijentima koji se određuju zamenom u diferencijalnu jednačinu. Kada se ovako nađe partikularno rešenje $y_p$, onda je opšte rešenje $y = y_h + y_p$.
Ovaj metod primenjuje se samo u slučaju kada je
:
\begin{enumerate}[label = \arabic*)]
	\item 
		$F(x) = e^{\alpha x} Pm(x)$ gde je $Pm(x)$ - polinom stepena $m$
		\begin{enumerate}[label = \alph*)]
			\item  $\alpha$ nije koren karakteristične jednačine $y_p = e^{\alpha x} Qm(x)$
			\item  $\alpha$ je koren reda $k$, $k \geq 1$ karakteristične jednačine $y_p = x^k e^{\alpha x} Qm(x)$
		\end{enumerate}
		Koeficijenti polinoma $Qm(x)$ nalaze se zamenom u nehomogenu diferencijalnu jednačinu, metodom neofređenih koeficijenata.
	
	\item 
		$F(x) = e^{\alpha x} (Pm_1(x)\cos \beta x + Pm_2???(x)\sin \beta x)$
		\begin{enumerate}[label = \alph*)]
			\item $\alpha \pm i \beta$ nije koren karakteristične jednačine $y_p = e^{\alpha x} (Qm(x)  \cos \beta x + Rm(x) \sin \beta x) \quad dgQm dgRm = max{m_1, m_2}$
			\item  $\alpha \pm i \beta$ je koren reda $k$, $k \geq 1$ karakteristične jednačine $y_p = x^k e^{\alpha x} (Qm(x)  \cos \beta x + Rm(x) \sin \beta x) \quad dgQm dgRm = max{m_1, m_2}$
		\end{enumerate}
		koeficijenti polinoma $Qm(x), \, Rm(x)$ nalaze se zamenom u nehomogenu diferencijalnu jednačinu metodom neodređenih koeficijenata
\end{enumerate}
	\section{Kombinatorika}
\subsection{Osnovni kombinatorni principi}
\begin{enumerate}[label=\textbf{\arabic*.)}]
	\item \textbf{Princip jednakosti}\\
		Ako između dva konačna skupa $A$ i $B$ postoji bijekcija $f:A\to B$ tada skupovi $A$ i $B$ imaju isti broj elemenata, tj. $$|A|=|B|$$
	\item \textbf{Princip zbira}
		Ako su $A$ i $B$ konačni, disjunktni skupovi, tada je $$|A\cup B|=|A|+|B|$$
		\textbf{Napomena:} Ili treba da asocira na zbir. Dakle kada su neka dva elementa u odnosu $\lor$ koristi se princip zbir
	\item \textbf{Princip proizvoda}
		Ako su $A$ i $B$ konačni skupovi, tada je $$|A\times B|=|A||B|$$
\end{enumerate}

\subsection{Permutacije, varijacije i kombinacije}
\begin{definition}
	\textbf{Permutacija} skupa $x_n = \{x_1, x_2, \ldots, x_n\}$ je bilo koja uređena $n$-torka različitih elemenata tog skupa
\end{definition}
\begin{example}
	Permutacije skupa $x_n = {1, 2, 3}$ su:
	\begin{align*}
		\begin{aligned}
		(1,2,3)\\
		(2,3,1)\\
		(3,1,2)\\
		\end{aligned}
		\quad \quad
		\begin{aligned}
		(3,2,1)\\
		(2,1,3)\\
		(1,3,2)\\
		\end{aligned}
	\end{align*}
	jer na prvom mestu može da bude bilo koji od članova skupa (3 mogućnosti), na drugom mestu mogu biti svi elementi osim jednog koji je na prvom mestu (2 mogućnosti), i na poslednjem samo jedan preostali element (1 mogućnost).\\
	Ukupno mogućnosti: $3*2*1=3!=6$	
\end{example}
\begin{theorem}
	\textbf{Broj permutacija} skupa od $n$ elemenata je $$n!$$
\end{theorem}

\begin{definition}
	\textbf{Varijacija} $k$-te klase skupa $x_n$ je bilo koja uređena $k$-torka različirih elemenata iz skupa $x_n$
\end{definition}
\begin{example}
	Varijacije 3. klase skupa $x_n = {a,b,c,d}$ su:
	\begin{align*}
		\begin{aligned}
		(a,b,c)\\
		(a,b,d)\\
		(a,c,b)\\
		(a,c,d)\\
		(a,d,b)\\
		(a,d,c)\\
		\end{aligned}
		\quad \quad
		\begin{aligned}
		(b,a,c)\\
		(b,a,d)\\
		(b,c,a)\\
		(b,c,d)\\
		(b,d,a)\\
		(b,d,c)\\
		\end{aligned}
		\quad \quad
		\begin{aligned}
		(c,a,b)\\
		(c,a,d)\\
		(c,b,a)\\
		(c,b,d)\\
		(c,d,a)\\
		(c,d,b)\\
		\end{aligned}
		\quad \quad
		\begin{aligned}
		(d,a,b)\\
		(d,a,c)\\
		(d,b,a)\\
		(d,b,c)\\
		(d,c,a)\\
		(d,c,b)\\
		\end{aligned}
	\end{align*}
	jer na prvom mestu može da bude bilo koji od članova skupa (4 mogućnosti), na drugom mestu mogu biti svi elementi osim jednoh koji je na prvom mestu (3 mogućnosti), i na poslednjem može biti neki od preostala dva (2 mogućnosti).\\
	Ukupno mogućnosti: $4*3*2=\frac{4!}{1!}=24$
\end{example}
\begin{theorem}
	\textbf{Broj varijacija} $k$-te klase skupa od $n$ elemenata je $$V_n^k = n(n-1)(n-2)\ldots(n-n+1) = \frac{n!}{(n-k)!}$$
\end{theorem}

\begin{definition}
	\textbf{Kombinacija} $k$-te klase skupa $x_n$ je bilo koji njegov podskup sa $k$ elemenata
\end{definition}
\begin{example}
	Kombinacije 2. klase skupa $x_n = {a,b,c,d}$ su:
	\begin{align*}
		\begin{aligned}
			\{a,b\}\\
			\{a,c\}\\
		\end{aligned}
		\quad 
		\begin{aligned}
			\{a,d\}\\
			\{b,c\}\\
		\end{aligned}
		\quad 
		\begin{aligned}
			\{b,d\}\\
			\{c,d\}\\
		\end{aligned}
	\end{align*}
	jer jedan član može da bude bilo koji (4 mogućnosti), a drugi može da bude neki od preostalih (3 mogućnosti), međutim pošto redosled nije bitan, a broj permutacija skupa gde je $n=2$  je $2! = 2*1 = 2$, ukupan broj kombinacija je 2 puta manji od broja varijacija 2. klase skupa $x_n$.\\
	Ukupno mogućnosti: $\frac{4*3}{2}=\frac{4!}{(4-2)!2!}=6$  \\
	Kombinacije 3. klase skupa $x_n$ su:
	\begin{align*}
		\{a,b,c\}\quad\{a,b,d\}\quad\{b,c,d\}\\
	\end{align*}
	jer jedan član može da bude bilo koji (4 mogućnosti), drugi može da bude neki od preostalih (3 mogućnosti) i poslednji može da bude neki od preostala 2 (2 mogućnosti), međutim pošto redosled nije bitan, a broj permutacija skupa gde je $n=3$ je $3! =6$, ukupan broj kombinacija je 6 puta manji od broja varijacija 3. klase skupa $x_n$.\\
	Ukupno mogućnosti: $\frac{4*3*2}{6}=\frac{4!}{(4-3)!3!}=4$
\end{example}
\begin{theorem}
	\textbf{Broj kombinacija} $k$-te klase skupa od $n$ elemenata je $$C_n^k=\binom nk = \frac{n(n-1)\ldots(n-(k-1)9}{k!} = \frac{n!}{(n-k)!k!}$$
\end{theorem}

\subsection{Varijacije i kombinacije sa ponavljanjem}
\begin{definition}
	\textbf{Varijacija sa ponavljanjem} $k$-te klase skupa $x_n$ je bilo koja uređena $k$-torka njegovih elemenata
\end{definition}
\begin{example}
	Varijacije sa ponavljanjem 2. klase skupa $x_n = \{a,b,c,d\}$ su:
	\begin{align*}
		\begin{aligned}
			(a,a)\\
			(a,b)\\
			(a,c)\\
			(a,d)\\
		\end{aligned}
		\quad \quad
		\begin{aligned}
			(b,a)\\
			(b,b)\\
			(b,c)\\
			(b,d)\\
		\end{aligned}
		\quad \quad
		\begin{aligned}
			(c,a)\\
			(c,b)\\
			(c,c)\\
			(c,d)\\
		\end{aligned}
		\quad \quad
		\begin{aligned}
			(d,a)\\
			(d,b)\\
			(d,c)\\
			(d,d)\\
		\end{aligned}
	\end{align*}
	jer da prvom mestu može biti bilo koji element (4 mogućnosti), a i na drugom mestu može biti bilo koji element jer su ponavljanja dozvoljena (4 mogućnosti).\\
	Ukupno mogućnosti: $4*4 = 4^2 = 16$
\end{example}
\begin{theorem}
	\textbf{Broj varijacija sa ponavljanjem} $k$-te klase skupa od $n$ elemenata je $$V_n^k = n^k$$
\end{theorem}

\begin{definition}
	Dato je ukupno $n$ objekata $k$ različitih tipova, $n_i(i=\overline{1,k})$ je broj ponavljanja $i$-tog tipa objekta, $n_1+ n_2+ \ldots+ n_k = n$. \textbf{Permutacija sa ponavljanjem} navedene familije je svaki raspored tih elemenata.
\end{definition}
\begin{example}
	Neke od permutacija sa ponavljanjem familije $a,a,b,b,b,c$ su:
	$$aabbbc,\; baabbc,\; baabcb,\;\ldots$$
	Ukupan broj permutacija sa ponavljanjem je, slično kao kod kombinacija, jednak broju permutacija podeljenom sa brojem istih elemenata svakog tipa, jer se ne može odrediti razlika između dve permutacije gde su zamenjena mesta elementima istog tipa.
	Ukupan broj mogućnosti: $\frac{6!}{2!3!1!}=60$
\end{example}
\begin{theorem}
	\textbf{Broj permutacija sa ponavljanjem} date familije je $$P_{n_1, n_2, \ldots, n_k} = \frac{n!}{n_1!n_2!\ldots n_k!}$$
\end{theorem}

\begin{definition}
	\textbf{Kombinacija $k$ te klase sa ponavljanjem} skupa $x_n$ je bilo koja familija sastavljena od tačno $k$ ne obavezno različitih elemenata skupa $x_n$
\end{definition}
\begin{example}
	Kombinacije sa ponavljanjem 2. klase skupa $x_n = {a,b,c,d}$ su:
	\begin{align*}
		\begin{aligned}
			(a,a)\\
			\\
			\\
			\\
		\end{aligned}
		\quad \quad
		\begin{aligned}
			(a,b)\\
			(b,b)\\
			\\
			\\
		\end{aligned}
		\quad \quad
		\begin{aligned}
			(a,c)\\
			(b,c)\\
			(c,c)\\
			\\
		\end{aligned}
		\quad \quad
		\begin{aligned}
			(a,d)\\
			(b,d)\\
			(c,d)\\
			(d,d)\\
		\end{aligned}
	\end{align*}
	Po principu jednakosti možemo naći bijekciju skupa $x_n$ koju možemo lakše da prebrojimo. Jedan način koji dosta olakšava ovaj problem je da zamislimo da postoje 4 polja (jer postoje 4 elementa u skupu $x_n$) koja su odvojena pomoću 3 crtice. Pošto su kombinacije 2. klase, imamo i 2 kružića koja treba da se rasporede u polja (s tim da u isto polje može da se stavi i više kružića, ili ni jedan). Na taj način dobijamo skup od 5 elemenata (3 crtice i 2 kružića). Jedino što preostaje je da rasporedimo crtice i kružiće proizvoljno (jer ne postoje ograničenja). Broj mogućnosti je sada odabir 2 mesta od 5 mogućih za kružiće dok se ostala popunjavaju crticama (ili obrnuto). Naravno, isti princip važi i ako je broj klasa veći od broja elemenata skupa (broj kružića veći od broj crtica)
\end{example}
\begin{definition}
	Reprezentacija prirodnoh broja $n$ u obliku $a_1+a_2+\ldots+a_k=n$ gde su $a_1, a_2, \ldots, a_k \in \mathbb{N}$ naziva se \textbf{podela} ili \textbf{razbijanje} tog broja, ili preciznije \textbf{$k$-podela}
\end{definition}

\begin{definition}
	\textbf{Kompozicija} broja $n$ je bilo koja uređena podela, tj. podela kod koje je poredak bitan.
\end{definition}

\begin{definition}
	\textbf{Particija} broja $n$ je bilo koja neuređena podela, tj. podela kod koje poredak sabiraka nije bitan.
\end{definition}
\begin{example}
	Particije i kompozicije broja $n=4$ su:
	\begin{align*}
		\begin{gathered}
			\text{Kompozicije:}\\
			4\\
			1+3,\;3+1,\;2+2\\
			1+1+2,\;1+2+1,\;2+1+1\\
			1+1+1+1\\
		\end{gathered}
		\quad \quad \quad
		\begin{gathered}
			\text{Particije:}\\
			4\\
			1+3,\;2+2\\
			1+1+2\\
			1+1+1+1\\
		\end{gathered}
	\end{align*}
\end{example}

\begin{example}
	Kompozicije broja $n=7$ koje imaju $k=4$ sabiraka mogu se prebrojati pomoću crtica i kružića. Zamislimo 7 kružića i između njih 6 polja, ako u neko polje stavimo crticu to je kao da smo stavili plus na to mesto. Samim tim, pošto želimo da odredimo broj kompozicija za 4 sabirka stavljam 3 crtice. Broj načina na koje ovo može da se uradi je jednak odabiru 3 mesta od 6.
\end{example}

\begin{theorem}
	\begin{enumerate}
		\item
			Broj kompozicija broja $n$ koje imaju $k$ sabiraka je $\binom{n-1}{k-1}$
		\item
			Uupan broj svih kompozicija broja $n$ je $2^{n-1}$
	\end{enumerate}
\end{theorem}

\begin{proof}
	\begin{enumerate}
		\item
			Postoji $n$ kružića $\implies$ postoji $(n-1)$ mesta na koja mogu da se stave crtice, kojih ima $(k-1)$. Samim tim ukupan broj mogućnosti je $\binom{n-1}{k-1}$ 
		\item
			Ukupan broj kompozicija se može dobiti na dva načina:
			\begin{enumerate}[label = \alph*)]
				\item 
					U svakom polju može da se pojavi crtica, ili ne mora, samim tim imamo 2 mogućnosti za svako polje (kojih ima $(n-1)$) pa je ukupan broj kompozicija $2^{(n-1)}$
				\item
					Ukupan broj kompozicija je $\sum_{k=1}{n} \binom{n-1}{k-1} = \sum_{i=0}{n-1} = 2^{n-1}$
			\end{enumerate}
	\end{enumerate}
\end{proof}
	\section{Bulova algebra}
\begin{definition}
	Skup $B$ sa najmanje 2 elementa na kome su definisane 2 binarne operacije $\land$ i $\lor$ je \textbf{Bulova algebra} ako važe sledeće aksiome:
	\begin{enumerate}[label=$B_{\arabic *}$: ]
		\item 
			Komutativnost 
			\begin{align*}
				(\forall a, b \in B) \; a\lor b &= b \lor a\\
				a\land b &= b\land a
			\end{align*}
		\item
			Asocijativnost
			\begin{align*}
				(\forall a, b, c \in B) \; (a\lor b) \lor c &= a\lor (b\lor c)\\
				(a\land b)\land c &= a \land (b\land c)
			\end{align*}
		\item
			Distributivnost
			\begin{align*}
				(\forall a, b, c \in B) \; a\lor (b\land c) &= (a\lor b)\land (a\lor c)\\
				a\land (b \lor c) &= (a \land b) \lor (a \land c)
			\end{align*}
		\item
			Egzistencija neutralnoh elementa
			\begin{align*}
				&(\exists 0 \in B)(\forall x \in B) \; x\lor 0 = x\\
				&(\exists 1 \in B)(\forall x \in B) \; x\land 1 = x
			\end{align*}
		\item
			Egzistencija komplemnta
			\begin{align*}
				(\forall x \in B)(\exists \overline{x} \in B)\; x\land \overline{x} &= 0\\
				x \lor \overline{x} &= 1
			\end{align*}
	\end{enumerate}
	Bulova algebra se označava sa $B=(B, \lor, \land)$ ili $B = (B, \lor, \land, \neg)$.\\
	\textbf{Napomena:} Ovaj sistem aksioma nije minimalan.
\end{definition}

\subsection{Primeri Bulove algebre}

\subsubsection{Binarna (dvočlana) Bulova algebra}
	$B = \{ 0, 1 \}$ i operacije $\land, \lor, \neg$ su definisane tablicama:
	\begin{table}[h!]
	\centering
	\begin{tabular}{|c|c|c|c|c|}
		\hline
		$p$ & $q$ & $\overline{p}$ & $p\land q$ & $p\lor q$ \\ \hline
		0 & 0 & 1 & 0        & 0       \\ \hline
		0 & 1 & 1 & 0        & 1       \\ \hline
		1 & 0 & 0 & 0        & 1       \\ \hline
		1 & 1 & 0 & 1        & 1       \\ \hline
	\end{tabular}
	\end{table}
\subsubsection{Algebra skupova}
	\begin{definition}
		Algebra $(\powerset{u}, \cup, \cap)$ gde je $\powerset{I}$ partitivni skup skupa $I\; (I\centernot = \emptyset)$, $\cup$ unija, $\cap$ presek i $^c$ komplement je Bulova algebra.\\
	\end{definition}
	Ovo sledi iz osobina koje važe za uniju i presek skupova:
	\begin{enumerate}[label = \arabic*.]
		\item $A \cup B = B \cup A \quad A \cap B = B \cap A$
		\item $(A\cup B) \cup C = A \cup (B \cup C) \quad (A\cap B) \cap C = A \cap (B \cap C)$
		\item $A \cup (B \cap C) = (A \cup B) \cap (A \cup C) \quad A \cap (B \cup C) = (A \cap B) \cup (A \cap C)$
		\item $A \cup \emptyset = A \quad A \cap I = A$
		\item $A \cup A^c = I \quad A \cap A^c = \emptyset$
	\end{enumerate}
	\begin{theorem}
		Za svaku konačnu Bulovu algebru $V=(V, \lor, \land)$ postoji skup $I$ i bijekcija $f:B \to \powerset{I}$ tako da važe relacije:
		\begin{align*}
			f(x \lor y) = f(x) \cup f(y)\\
			f(x \land y) = f(x) \cap f(y)
		\end{align*}
	\end{theorem}
	\begin{corollary}
		\begin{itemize}
			\item Stounova teorema tvrdi da je svaka konačna Bulova algebra izomorfna nekoj algebri skupova $(\powerset{I}, \cup, \cap)$
			\item Ova teorema omogućava da se u dokazivanju rezultata Bulove algebre koriste metode iz teorije skupova
			\item Bulova algebra može imati najviše $2^n$ elemenata, gde je $n \in \mathbb{N}$, jer je $\left| \powerset{I}\right| = 2^n$, za $|I|=n$.
		\end{itemize}
	\end{corollary}
\subsection{Princip dualnosti}
\begin{definition}
	Ako je neka jednakost teorema Bulove algebre, tada zamenom operacija $\land$ i $\lor$ i elemenata $0$ i $1$ u toj relaciji dolazimo do tačne jednakosti. Ta jednakost naziva se \textbf{dualna teorema} date teoreme
\end{definition}

\subsection{Neke važnije teoreme Bulove algebre}
\begin{theorem}
	\begin{enumerate}[label = \textbf{\arabic*.)}]
		\item Idempotentnost $(\forall a \in B) \; a \lor a = a \quad a\land a = a$
		\item $(\forall a \in B)\; a \lor q = q \quad a \land 0 = 0$
		\item Apsorpcija $(\forall a, b \in B)\; a \lor (a \land b) = a \quad a \land (a \lor b) = a$
		\item Involucija $(\forall a \in B)\; \overline{(\overline{a})} = a$
		\item Za neutralne elemente $0$ i $1$ važi $\overline{0} = 1 \quad \overline{1} = 0$
		\item U Bulovoj algebri elementi $0$ i $1$ su jedinstveni
		\item U Bulovoj algebri svaki element $a \in B$ ima jedinstveni komplement $\overline{a}$
		\item De Morganovi zakoni $(\forall a, b \in B)\; \overline{a \land b} = \overline{a} \lor \overline{v} \quad \overline{a \lor b} = \overline{a} \land \overline{b}$
	\end{enumerate}
\end{theorem}
\begin{proof}
	\begin{enumerate}[label = \textbf{\arabic*.)}]
		\item
		\item 
			\begin{align*}
				&a \lor 1 = 1\\
				&a \lor 1 = (a \lor 1) &(B_4)\\
				&= 1 \land (a \lor 1) &(B_1)\\
				&= (a \lor \overline{a}) \land (a \lor q) &(B_5)\\
				&= a \lor (\overline{a} \land 1) &(B_3)\\
				&= a \lor \overline{a} &(B_4)\\
				&= 1 &(B_5)
			\end{align*}
		\item
		\item
		\item
	\end{enumerate} 
	
		?????????????????????????????????????????
\end{proof}

\subsection{Binarna Bulova algebra}
\subsubsection{Bulovi izrazi u binarnoh Bulovoj algebri}
	$B=(\{0, 1\}, \lor, \land, \neg)$
	\begin{definition}
		U skupu $\{0, 1\}$ definišemo izraz:
		\begin{align*}
			x^a = \left\{
			\begin{aligned}
				&\overline{x}, \; &a = 0\\
				&x, \; &a = 1
			\end{aligned}
			\right.
		\end{align*}
	\end{definition}
	\begin{definition}
		Bulovi izrazi su:
		\begin{enumerate}[label = \arabic*)]
			\item
				Bulove konstante $0$, $1$\\
				Bulove promenljive $x$, $y$, $z$,...
			\item
				Ako su $A$ i $V$ Bulovi izrazi, onda su i $A \lor B$, $A \land B$, $\overline{A}$ Bulovi izrazi
			\item
				Bulovi izrazi se mogu dobiti konačnim brojem primena tačaka 1) i 2)
		\end{enumerate}
	\end{definition}
	\begin{definition}
		\textbf{Elementarna konjukcija (EK)} promenljivih $x_1, x_2,\ldots, x_n$ je Bulov izraz oblika:
		$$x_{i_1}^{\alpha_{i_1}} \land x_{i_2}^{\alpha_{i_2}} \land \ldots \land x_{i_m}^{\alpha_{i_m}}$$
		gde $\{i_1, i_2, \ldots, i_n\} \subset \{1, 2, \ldots, n\}$ i  $\alpha_{i_1}, \alpha_{i_2}, \ldots, \alpha_{i_m} \in {0,1}$
	\end{definition}
	\begin{definition}
		\textbf{Kanonička elementarna konjukcija (KEK)} promenljivih $x_1, x_2,\ldots, x_n$ je Bulov izraz oblika:
		$$x_1^{\alpha_1} \land x_2^{\alpha_2} \land \ldots \land x_n^{\alpha_n}$$
		gde $\alpha_{i_1}, \alpha_{i_2}, \ldots, \alpha_{i_n} \in {0,1}$
		\textbf{Napomena:} učestvuju sve promenljive.
	\end{definition}
	\begin{definition}
		\textbf{Disjunktivna forma (DF)} je Bulov izraz oblika:
		$$\mathrm{EK_1} \lor \mathrm{EK_2} \lor \ldots \lor \mathrm{EK_m}$$
		gde su $\mathrm{EK_1}, \mathrm{EK_2}, \ldots, \mathrm{EK_m}$ elementarne konjukcije. 
	\end{definition}
	\begin{definition}
		\textbf{Savršena disjunktivna normalna forma (SDNF)} u odnosu na promenljive $x_1, x_2,\ldots, x_n$ je Bulov izraz oblika:
		$$\mathrm{KEK_1} \lor \mathrm{KEK_2} \lor \ldots \lor \mathrm{KEK_m}$$
		gde su $\mathrm{KEK_1}, \mathrm{KEK_2}, \ldots, \mathrm{KEK_m}$ kanoničke elementarne konjukcije. 
	\end{definition}
	
	\begin{definition}
		\textbf{Elementarna disjunkcija (ED)} promenljivih $x_1, x_2,\ldots, x_n$ je Bulov izraz oblika:
		$$x_{i_1}^{\alpha_{i_1}} \lor x_{i_2}^{\alpha_{i_2}} \lor \ldots \lor x_{i_m}^{\alpha_{i_m}}$$
		gde $\{i_1, i_2, \ldots, i_n\} \subset \{1, 2, \ldots, n\}$ i  $\alpha_{i_1}, \alpha_{i_2}, \ldots, \alpha_{i_m} \in {0,1}$
	\end{definition}
	\begin{definition}
		\textbf{Kanonička elementarna disjunkcija (KED)} promenljivih $x_1, x_2,\ldots, x_n$ je Bulov izraz oblika:
		$$x_1^{\alpha_1} \lor x_2^{\alpha_2} \lor \ldots \lor x_n^{\alpha_n}$$
		gde $\alpha_{i_1}, \alpha_{i_2}, \ldots, \alpha_{i_n} \in {0,1}$\\
		\textbf{Napomena:} učestvuju sve promenljive.
	\end{definition}
	\begin{definition}
		\textbf{Konjuktivna forma (KF)} je Bulov izraz oblika:
		$$\mathrm{ED_1} \land \mathrm{ED_2} \land \ldots \land \mathrm{ED_m}$$
		gde su $\mathrm{ED_1}, \mathrm{ED_2}, \ldots, \mathrm{ED_m}$ elementarne disjunkcije. 
	\end{definition}
	\begin{definition}
		\textbf{Savršena konjuktivna normalna forma (SKNF)} u odnosu na promenljive $x_1, x_2,\ldots, x_n$ je Bulov izraz oblika:
		$$\mathrm{KED_1} \land \mathrm{KED_2} \land \ldots \land \mathrm{KED_m}$$
		gde su $\mathrm{KED_1}, \mathrm{KED_2}, \ldots, \mathrm{KED_m}$ kanoničke elementarne disjunkcije. 
	\end{definition}
	
\subsection{Bulove funkcije}
\begin{definition}
	Preslikavanje $f: \{0, 1\}^n \to \{0, 1\}$ naziva se Bulova funkcija.\\
	\textbf{Napomena:} Bulove funkcije najčešće se zadaju preko Bulovih izraza ili pomoću zablica.
\end{definition}
\begin{theorem}
	Broj različitih Bulovih funkcija $f: \{0, 1\}^n \to \{0, 1\}$ sa $n$ promenljivih je $2^{2^n}$
\end{theorem}
\begin{proof}
	Broj mesta u koloni jednak je broju redova. Pošto postoji $n$ promenljivih i svaka može da ima neku od vrednosti $0$ ili $1$ broj redova je $2^n$. U svakom redu funkcija može da ima neku od vrednosti $0$ ili $1$, što su dve mogućnosti, pa je ukupan broj funkcija jednak $\underbrace{2\cdot2\cdot2\cdots2}_{2^n} = 2^{2^n}$.
	\begin{table}[h!]
	\centering
	\begin{tabular}{|c|c|c|c|c|}
		\hline
		$x_1$ & $x_2$ & $\cdots$ & $x_n$ & $f(x_1, x_2, \ldots, x_n)$ \\ \hline
		$0$ & $0$ & $\cdots$ & $0$ & $y_1$ \\ \hline
		$0$ & $0$ & $\cdots$ & $1$ & $y_2$ \\ \hline
		$\vdots$ & $\vdots$ & $\vdots$ & $\vdots$ & $\vdots$ \\ \hline
		$1$ & $1$ & $\cdots$ & $1$ & $y_{2^n}$ \\ \hline
	\end{tabular}
	\end{table}
\end{proof}
\begin{theorem}
	Svaka Bulova funkcija zadata pomoću Bulovog izraza može se izraziti pomoću tablice.
\end{theorem}
\begin{theorem}
	Za svaku Bulovu funkciju, izuzev funkcije koja je identički jednaka nuli važi:
	$$f(x_1, x_2, \ldots, x_2) = \lor \left[ f(\alpha_1, \alpha_2, \ldots, \alpha_n) \land x_1^{\alpha_1} \land x_2^{\alpha_2} \land \ldots \land x_n^{\alpha_n} \right]$$
	gde je $\alpha_1, \alpha_2, \ldots, \alpha_n \in {0, 1}$.
\end{theorem}
\begin{theorem}Za svaku Bulovu funkciju, izuzev funkcije koja je identički jednaka nuli važi:
	$$f(x_1, x_2, \ldots, x_2) = \land \left[ f(\alpha_1, \alpha_2, \ldots, \alpha_n) \lor x_1^{\alpha_1} \lor x_2^{\alpha_2} \lor \ldots \lor x_n^{\alpha_n} \right]$$
	gde je $\alpha_1, \alpha_2, \ldots, \alpha_n \in {0, 1}$.
\end{theorem}

\subsection{Baze skupa Bulovih funkcija}
\begin{definition}
	Skup Bulovih funkcija $\mathbb{F}$ je \textbf{generatorski skup} skupa svih Bulovih funkcija ako se pomoću funkcija iz $\mathbb{F}$ mogu izraziti sve Bulove funkcije.
\end{definition}
\begin{example}
	Skup $\{\land, \lor, \neg\}$ i svaki njegov nadskup su generatorski skupovi
\end{example}
\begin{definition}
	Skup Bulovih funkcija $\mathbb{U}$ je \textbf{baza} skupa svih Bulovih funkcija ako je:
	\begin{enumerate}[label = \arabic*)]
		\item $\mathbb{U}$ generatorski skup skupa svih Bulovih funkcija
		\item Nijedan pravi podskup skupa $\mathbb{U}$ nije generatorski skup
	\end{enumerate}
\end{definition}
\begin{example}
	Baze skupa svih Bulovih funkcija su:
	$$\{\lor, \neg\} \quad \{\land, \neg\} \quad \{\Rightarrow, \neg\} \quad \{\downarrow\} \quad \{\uparrow\}$$
\end{example}
\begin{proof}
	Dovoljno je za skup $\{ \downarrow \}$ dokazati da je generatorski skup.
	\begin{gather*}
		\overline{x} = x \downarrow y = \overline{x \lor x} = x \downarrow\\
		x \lor y = \overline{\overline{x \lor y}} = \overline{x \downarrow y} = (x \downarrow y) \downarrow (x \downarrow y)\\
		x \land y = \overline{\overline{x \land y}} = \overline{\overline{x} \lor \overline{y}} = \overline{x} \downarrow \overline{y} = (x \downarrow x) \downarrow (y \downarrow y)
	\end{gather*}
\end{proof}
\begin{definition}
	Iskazna formula koja za sve vrednosti iskaznih slova koja se u njoj pojavljuju dobija vrednost $1$ naziba se \textbf{tautologija}.
\end{definition}
\end{document}