\section{Diferencijalne jednačine}
- Principijalno ne znamo da ih rešavamo\\
- Neke i možemo da rešimo (obratiti pažnju na klasifikaciju)
\subsection{Obične diferencijalne jednačine}
\begin{definition}
	Implicitan izraz $f(x,y,y',\ldots,y^{(n)})=0$ gde je $y:(a,b)\to \mathbb{R}$ nepoznata, $n$ puta diferencijabilna funkcija nezavisno promenljive $x$, naziva se \textbf{obična diferencijabilna jednačina} reda $n$ ako u njoj efektivno učestvuje izvod $y^{(n)}$.\\
	\textbf{Napomena:} nije obavezno pojavljivanje svih članova, ali je obavezno pojavljivanje $n$-tog izvoda.
\end{definition}
\begin{definition}
	\textbf{Rešenje} diferencijalne jednačine na intervalu $I$ je svaka funkcija $y$ definisana na $I$ koja jednačinu svodi na identitet.\\
\end{definition}
\begin{definition}
	\textbf{Opšte rešenje} diferencijalne jednačine $n$-tog reda je svaka funkcija $y$ definisana sa $G(x,y,c_1,c_2,\ldots,c_n)$ gde su $c_1,c_2,\ldots,c_n$ proizvoljne konstante iz $\overline{\mathbb{R}}$, tako da:
	\begin{enumerate}[label=\arabic*)]
		\item
			$y$ jeste rešenje jednačine
		\item
			Polazna jednačine se može dobiti iz izraza $G$
	\end{enumerate}
	\textbf{Napomena:} Opšte rešenje ne sadrži sva rešenja diferencijalne jednačine
\end{definition}
\begin{definition}
	\textbf{Partikularno rešenje} diferencijalne jednačine $n$-tog reda je svako njeno rešenje koje je obuhvaćeno opštim rešenjem, tj. koje se može dobiti iz opšteg rešenja za neke konkretne vrednosti konstanti.
\end{definition}
\begin{definition}
	\textbf{Singularno rešenje} diferencijalne jednačine $n$-tog reda je svako rešenje koje nije obuhvaćeno opštim rešenjem.
\end{definition}
\begin{definition}
	Partikularno rešenje diferencijalne jednačine $n$-tog reda određeno uslovima:
	$$\left.
	\begin{aligned}
		y(x_0) &= y_0\\
		y'(x_0) &= y_1\\
		\vdots\\
		y^{(n-1)}(x_0) &= y_{n-1}\\
		y^{(n)}(x_0) &= y_n
	\end{aligned}
	\right\} \text{Košijevi (početni) uslovi}
	$$
	naziva se \textbf{Košijevo rešenje} diferencijalne jednačine za početne uslove.\\
	\textbf{Napomena:} Ovaj problem drugačije se naziva Košijev problem ili problem početnih uslova. Rešava se $n$ jednačina ($n$ uslova) sa $n$ nepoznatih ($n$ konstanti). Ukoliko je problem korektno zadat onda ima jedinstveno rešenje.
\end{definition}


\subsection{Diferencijalne jednačine prvog reda}


\subsubsection{Diferencijalne jednačina kod kod kojih promenljive mogu da se razdvoje}
\textbf{Opšti izraz:}
\begin{gather*}
	f(x)\, \mathrm{d}x +g(y)\, \mathrm{d}y = 0\\
	y' = \varphi(x)\xi(y)
\end{gather*}
gde su funkcije $f$, $g$, $\varphi$ i $\xi$ definisane i neprekidne na nekom intervalu $I$ na kom se traži rešenje ove jednačine.
\textbf{Opšte rešenje:}
$$\int f(x) \, \mathrm{d}x + \int g(y) \, \mathrm{d}y = c$$
\textbf{Partikularno rešenje Košijevog problema:}
$$\int \limits^x_{x_0} f(x) \, \mathrm{d}x +\int \limits^y_{y_0} g(y) \, \mathrm{d}y = 0$$
\textbf{Napomena:} Na kraju rešavanja broj konstanti mora biti jednak redu diferencijalne jednačine.


\subsubsection{Diferencijalne jednačine oblika $y'=g(ax+by)$}
\textbf{Opšti izraz:}
$$y'=g(ax+by)$$
gde su funkcije $y$ i $g$ definisane i neprekidne na nekom intervalu $I$ na kom se traži rešenje ove jednačine.
\textbf{Opšte rešenje:}
\begin{enumerate}[label = \alph*)]
	\item
		$b=0 \implies$ ovo je jednačina kod koje promenljive mogu da se razdvoje
	\item
		$b \centernot = 0 \implies$ u ovoj jednačini se uvodi smena zavisno promenljive $y$:
		\begin{align*}
			\begin{aligned}[c]
				ax+by=z\\
				by = z-ax\\
				y = \frac{z-ax}{b}\\
				y' = \frac{z'-a}{b}\\	
				y' = g(z)\\	
			\end{aligned}
			\quad \quad
			\begin{aligned}[c]
				\frac{z'-a}{b} = g(z)\\
				z' = bg(z)+a\\
				\frac{\mathrm{d}z}{\mathrm{d}x} = bg(z)+a\\
				\mathrm{d}x = \frac{\mathrm{d}z}{bg(z)+a}\\
				\int\frac{\mathrm{d}z}{bg(z)+a} = \int \mathrm{d}x + c\\
			\end{aligned}\\		
		\end{align*}
		$$\underbrace{\int\frac{\mathrm{d}z}{bg(z)+a} = x + c}_{\text{Rešavanjem dobija se opšte rešenje}}$$
\end{enumerate}

\subsubsection{Homogena diferencijalna jednačina prvog reda}
\textbf{Opšti izraz:}
$$y' = f\left(\frac{y}{x}\right)$$
gde je funkcija $f$ definisana na intervalu $I$ na kom se traži rešenje ove jednačine. I $f(u)\centernot \equiv u \implies$ nije identičko preslikavanje\\
\textbf{Opšte rešenje:}
\begin{align*}
	\begin{aligned}[c]
		z = \frac{y}{x}\\
		y = xz\\
		y' = z+z'x\\
		y' = f\left(\frac{y}{x}\right)\\
	\end{aligned}
	\quad \quad
	\begin{aligned}[c]
		z'x+z = f(z)\\
		\frac{\mathrm{d}z}{\mathrm{d}x}x = f(z)-z\\
		\frac{\mathrm{d}z}{f(z)-z} = \frac{\mathrm{d}x}{x}\\
		\int \frac{\mathrm{d}z}{f(z)-z} = \int \frac{\mathrm{d}x}{x}+c\\
	\end{aligned}\\	
\end{align*}
$$\underbrace{\int \frac{\mathrm{d}z}{f(z)-z} = \ln|x|+c}_{\text{Rešavanjem dobija se opšte rešenje}}$$


\subsubsection{Diferencijalna jednačina prvog reda oblika $y' = f\left(\frac{a_1x+b_1y+c_1}{a_2x+b_2y+c_2}\right)$}
\textbf{Opšti izraz:}
$$y' = f\left(\frac{a_1x+b_1y+c_1}{a_2x+b_2y+c_2}\right)$$
gde je funkcija $f$ definisana i neprekidna na nekom intervalu $I$ na kom se traži rešenje ove jednačine
\textbf{Opšte rešenje:}
\begin{enumerate}[label = \alph*)]
	\item
		$a_1 = a_2 = b_1 = b_2 = 0$ - Nije poželjno da istovremeno budu jednake nuli.
	\item
		$a_1=a_2=0 \lor b_1=b_2=0 \lor a_1=b_2=0 \lor b_1=a_2=0 \implies$ Diferencijalna jednačina kod koje promenljive mogu da se razdvoje.
	\item
		$c_1=c_2=0 \implies$ Homogena diferencijalna jednačina.
	\item
		$\left|
		\begin{matrix}
			a_1 & b_1\\
			a_2 & b_2
		\end{matrix}
		\right| = 0 \implies \frac{a_1}{a_2}=\frac{b_1}{b_2}\implies$ Koeficijenti $ a_1$, $a_2$, $b_1$ i $b_2$ su proporcionalni $\implies$ uvođenjem smene $a_1x+b_1y=z$ svodi se na diferencijalnu jednačinu oblika $y'=g(a_1x+b_1y)$ koju znamo da rešimo.
	\item
		$\left|
		\begin{matrix}
			a_1 & b_1\\
			a_2 & b_2
		\end{matrix}
		\right| \centernot = 0 \implies \frac{a_1}{a_2}\centernot=\frac{b_1}{b_2}\implies$Koeficijenti $ a_1$, $a_2$, $b_1$ i $b_2$ nisu proporcionalni $\implies$ smetaju $c_1$ i $c_2$, pa se uvode dve smene:
		\begin{enumerate}[label = \arabic*)]
			\item
				smena nezavisno promenljive $x=u+\alpha$
			\item
				smena zavisno promenljive $y=v+\beta$		
		\end{enumerate}				
		gde su $\alpha$ i $\beta$ konstante, rešenja sistema linearnih jednačina:
		$$\left.
		\begin{aligned}
			a_1\alpha+b_1\beta +c_1=0\\
			a_2\alpha+b_2\beta +c_2=0
		\end{aligned}
		\right\}a_i,\, b_i,\, c_i \text{ su koeficijenti iz polazne jednačine}$$
		$\det \centernot = 0\implies$ sistem ima jedinstveno rešenje, $\alpha$ i $\beta$ daju nove promenljive i jednačina se svodi na homogenu diferencijalnu jednačinu prvog reda.
\end{enumerate}

\subsubsection{Linearna diferencijalna jednačina prvog reda}
\textbf{Opšti izraz:}
$$y'+P(x)y=Q(x)$$
gde su $P$ i $Q$ funkcije definisane i neprekidne na nekom intervalu $I$ na kom se traži rešenje ove jednačine
\textbf{Opšte rešenje:}
\begin{enumerate}[label = \alph*)]
	\item 
		Homogena\\
		$Q(x) = 0 \implies$ promenljive mogu da se razdvoje
		\begin{align*}
			\begin{aligned}[c]
				-P(x)y=y'\\
				\frac{\mathrm{d}y}{y} = -P(x)\,\mathrm{d}x\\
			\end{aligned}
			\quad \quad
			\begin{aligned}[c]
				\int \frac{\mathrm{d}y}{y} = - \int P(x)\,\mathrm{d}x+c\\
				\ln|y| = -\int P(x)\mathrm{d}x+\ln c\\	
			\end{aligned}
		\end{align*}
		$$y = ce^{-\int P(x) \mathrm{d}x}$$
	\item
		Nehomogena\\
		\begin{align*}
			\begin{aligned}[c]
				y'+P(x)y=Q(x)\\
				y'e^{\int P(x)\mathrm{d}x} + yP(x)e^{\int P(x)\mathrm{d}x} = Q(x)e^{\int P(x)\mathrm{d}x}\\
				\left(ye^{\int P(x)\mathrm{d}x}\right)'=Q(x)e^{\int P(x)\mathrm{d}x}
			\end{aligned}
			\quad \quad
			\begin{aligned}[c]
				\int\left(ye^{\int P(x)\mathrm{d}x}\right)'=\int Q(x)e^{\int P(x)\mathrm{d}x}\\
				ye^{\int P(x)\mathrm{d}x} = \int Q(x)e^{\int P(x)\mathrm{d}x}\,\mathrm{d}x +c\\
			\end{aligned}
		\end{align*}
		$$y = e^{-\int P(x)\mathrm{d}x}\left(c+ \int Q(x)e^{\int P(x)\mathrm{d}x} \,\mathrm{d}x \right)$$
		Košijev problem, opšte rešenje za linearnu nehomogenu diferencijalnu jednačinu prvog reda:
		$$y = e^{-\int\limits^x_{x_0} P(x)\mathrm{d}x}\left(y_0+ \int\limits^x_{x_0} Q(x)e^{\int\limits^x_{x_0} P(x)\mathrm{d}x} \, \mathrm{d}x \right)$$
\end{enumerate}

\subsubsection{Bernulijeva diferencijalna jednačina}
\textbf{Opšti izraz:}
$$y'+P(x)y=Q(x)y^m$$
gde je $m \in \mathbb{R}\setminus \{0,1\}$, funkcije $P$ i $Q$ definisane i neprekidne na nekom intervalu $I$ na kom se rešava jednačina.\\
\textbf{Opšte rešenje:}
\begin{align*}
	\begin{aligned}[c]
		z = y^{1-m}\\
		z' = (1-m)y^{-m}y'\\
		y'+P(x)y=Q(x)y^m
	\end{aligned}
	\quad \quad
	\begin{aligned}[c]
		\frac{y'}{y^m}+P(x)y^{1-m}=Q(x)\\
		(1-m)y^{-m}y' + P(x)y^{1-m}=(1-m)Q(x)
	\end{aligned}
\end{align*}
$$z'+z\underbrace{P(x)}_{P_1(x)}=\underbrace{(1-m)Q(x)}_{Q_1(x)}$$ - linearna diferencijalna jednačina prvog reda

\subsubsection{Rikatijeva jednačina}
\textbf{Opšti izraz:}
$$y' = P(x)y^2+Q(x)y+R(x)$$
gde su funkcije $P$, $Q$ i $R$ definisane i neprekidne na nekom intervalu $I$ na kom se rešava jednačina.\\
\textbf{Opšte rešenje:}
Izaberimo takvu smenu da uklonimo $R(x)$ i dobijamo Bernulijevu diferencijalnu jednačinu za $m=2$\\
$y^2 \implies m= 2 \implies$ potrebno nam je jedno partikularno rešenje, od toga polazimo $(y_p)$\\
$y=y_p+\frac{1}{z} \implies$ smena zavisno promenljive\\
$y' = y'_p - \frac{z'}{z^2} \implies$ izvod nije nula\\
\begin{gather*}
	y'_p - \frac{z'}{z^2} = ((y_p)^2 + \frac{2y_p}{z} + \frac{1}{z^2}) P(x) + Q(x)(y_p + \frac{1}{z}) + R(x)\\
	y'_p - \frac{z'}{z^2} = y_p^2 P(x) + 2P(x)\frac{y_p}{z} + \frac{P(x)}{z^2} + Q(x)y_p + \frac{Q(x)}{z} + R(x)\\
	y'_p = P(x)y_p^2 + Q(x)y_p + R(x)\\
	z' = -2zP(x)y_p - P(x) - zQ(x) \implies\\
	z' + 2zP(x)y_pz+zQ(x)= -P(x)\\
	z' + z\underbrace{(2P(x)y_p+Q(x))}_{P_1} = \underbrace{-P(x)}_{Q_1}
\end{gather*} 
- linearna diferencijalna jednačina prvog reda

\subsection{Diferencijalne jednačine drugog reda}

\textbf{Opšti izraz:}
$$ F(x, y, y', y'') $$
gde je $F$ funkcija definisana i dva puta diferencijabilna na nekom intervalu $I$ na kom se jednačina rešava.

\subsubsection{Nepotpune diferencijalne jednačine drugog reda}
\begin{enumerate}[label = \textbf{\arabic*. slučaj}] 
	\item 
		$ F(y', y'') = 0 $ ili $ y'' = f(y') $\\
		- uvodi se smena zavisno promenljive $ p = y $\\
		\begin{align*}
			\begin{aligned}
				y' = p\\
				y'' = \frac{\dif p}{\dif x}\\
				y'' = f(p)\\
			\end{aligned}
			\quad \quad
			\begin{aligned}
			\frac{\dif p}{\dif x} = f(p)\\
			\int \frac{\dif p}{f(p)} = x + c \implies \varphi (x, p, c_1) = 0
			\end{aligned}
		\end{align*}
		Dolazi se do diferencijalne jednačine prvog reda $ \varphi (x, y, c_1) = 0 $, i na kraju rešavanjem te diferencijalne jednačine dolazi se do: $\psi (x, y, c_1, c_2) = 0$
	\item 
		$ F(x, y, y'') $\\
		- uvodi se smena zavisno promenljive $ p = y' $\\
		\begin{align*}
		\begin{aligned}
		y' = p\\
		y'' = \frac{\dif p}{\dif x}
		\end{aligned}
		\implies F(x, p, \frac{\dif p}{\dif x}) = 0
		\end{align*}
		Dolazi se do diferencijalne jednačine prvog reda $ \varphi (x, y', c_1) = 0 $, i na kraju rešavanjem te diferencijalne jednačine dolazi se do: $\psi (x, y, c_1, c_2) = 0$
	\item 
		$ F(y, y', y'') = 0 $\\
		- uvodi se smena zavisno promenljive $ y' = p $ (gubimo $ x $ tokom smene)\\
		\begin{align*}
		\begin{aligned}
		y' = p\\
		y'' = \frac{\dif p}{\dif x} \frac{\dif y}{\dif y} = y' \frac{\dif p}{\dif y} = p \frac{\dif p}{\dif y}\\
		\end{aligned}
		\implies F(y, p, p \frac{\dif p}{\dif x}) = 0
		\end{align*}
		Dolazi se do diferencijalne jednačine prvog reda $ \varphi (y, p, c_1) = 0 $, i na kraju rešavanjem te diferencijalne jednačine dolazi se do: $\psi (x, y, c_1, c_2) = 0$
\end{enumerate}

\subsection{Linearna diferencijalna jednačina $n$-tog reda}
\begin{definition}
	Diferencijalna jednačina oblika $y^{(n)} + f_1(x)y^{(n-1)} + \ldots + f_n(x)y = F(x)$ naziva se \textbf{linearna diferencijalna jednačina $n$-tog reda}. Pri tome podrazumevamo da su $f_i(x)$ i $F(x)$ definisane i neprekidne na posmatranom intervalu $I$.
	\begin{enumerate}[label = \arabic*.)]
		\item
			$F(x) \equiv 0$ - homogena jednačina
		\item 
			$F(x) \centernot \equiv 0$ - nehomogena jednačina
	\end{enumerate} 
	\Napomena ako su $f_i(x)$ gde je $i = \overline{1, n}$ konstante jednačina je sa konstantnim koeficijentima (u suprotnom je sa gunkcionalnim).
\end{definition}

\subsubsection{Homogena linearna diferencijalna jednačina}
	\begin{definition}
		$L^{n}(y) = y^{n} + d_1(x)y^{n-1} + \ldots + f_n(x)y = 0$ naziva se \textbf{linearni diferencijalni operator $n$-tog reda}
	\end{definition}
	Jedno rešenje (trivijalno) diferencijalne jednačine $L^{n}(y)$ je $y=0$, ali mi tražimo netrivijalna. Takođe se vidi: ako su $y_1(x)$ i $y_2(x)$ rešenja, tada je i $c_1y_1(x) + x_2 y_2(x)$ takođe rešenje date diferencijalne jednačine. Što se može uopštiti na proizvoljan broj rešenja. 
	\begin{gather}
		L^n (y_1) = 0 \land L^n(y_2) = 0\\
		L^n (y_1c_1 + y_2c_2) = c_1L^n(y_1) + x_2 L^n(y_2) = c_1\cdot 0 + c_2\cdot 0= 0+0 = 0
	\end{gather}
	\begin{definition}
		Funkcije $y_1(x), y_2(x), \ldots, y_n(x)$ definisane na intervalu $I$ mogu biti:
		\begin{enumerate}[label = \arabic*.)]
			\item 
				Linearno zavisne na $I$\\
				- ako postoje konstante $c_2, c_2, \ldots, c_n$ takve da $\sum_{k=0}^{n}c_k^2 \centernot 0 $ i da važi da je $\forall x \in I) c_1y_1(x) + c_2y_2(x) + \ldots + c_ny_n(x) = 0$
			\item 
				Linearno nezavisne na $I$\\
				- u suprotnom
		\end{enumerate}
	\Napomena u slučaju $n=2$ linearna nezavisnost se sodi na $\frac{y_1}{y_2} \centernot const$
	\end{definition}
	\begin{theorem}
		Ako su  $y_1(x), y_2(x), \ldots, y_n(x)$ rešenja jednačine $L^n(y) = 0$ i ako su funkcije  $y_1(x), y_2(x), \ldots, y_n(x)$ linearno nezavisne, tada je opšte rešenje date jednačine:
		$$ y =c_1y_1(x) + c_2y_2(x) + \ldots + c_ny_n(x) $$
		gde su $c1$ za $i = \overline{1, n}$ konstante
	\end{theorem}

\subsubsection{Homogena linearna diferencijalna jednačina drugog reda}
	\textbf{Opšti izraz: }
	$$ L^2(y) = 0 \; \lor \; y'' + P(x)y' + Q(x) y = 0 $$
	\begin{theorem}
		Liuvilova formula: Ako je $y_1(x)$ jedno netrivijalno partikularno rešenje linearne diferencijalne jednačine $y'' + P(x)y+ Q(x)y = 0$ tada je:
		$$y_2(x) = y_1(x) \int \frac{1}{y_1^2(x)} e^{-\int P(x) \dif x} \dif x$$
		takođe partikularno rešenje date jednačine linearno nezavisno od $y_1$.
	\end{theorem}
	\begin{proof}
		\begin{align*}
			\begin{gathered}
				y = y_1z\\
				y' = y'_1z + y_1 z'\\
				y'' = y''_1z + 2y'_1z' + z''y_1\\
				y''_1z+2y'_1z' + y_1z'' + P(x)y+_1z + P(x)y_1z' + Q(x)y_1z = 0\\
				(y''_1 + P(x)y'_1 + Q(x)y_1) + z'(2y'_1 + P(x)y_1) + z''y_1 = 0\\
				y''_1 + P(x)y'_1 + Q(x)y_1 = 0 \text{ partikularno, netrivijalno}\\
				z'' + z'(\frac{2y'_1}{y_1})+ P(x)) = 0\\
			\end{gathered}
			\quad \quad
			\begin{gathered}
				z' = x_1 e^{-\int (\frac{2y'_1}{y_1}+P(x)) \dif x} = \frac{\dif z}{\dif x}\\
				\frac{\dif z}{\dif x} = c_1 e^{-2\ln |y_1|-\int P(x) \dif x}\\
				\frac{\dif z}{\dif x} = c_1 e^{\ln \frac{1}{y_1^2}-\int P(x) \dif x}\\
				\frac{\dif z}{\dif x} = c_1 \frac{1}{y_1^2}e^{-\int P(x) \dif x}\\
				\dif z = \frac{c_1}{y_1^2} e^{-\int P(x) \dif x} \dif x\\
				z = \int \frac{c_1}{y_1^2} e^{-\int P(x) \dif x} \dif x + c_2\\
				y = y_1c_2 + c_1y_1 \int \frac{e^{- \int P(x) \dif x}}{y_1^2} \dif x\\
			\end{gathered}
		\end{align*}
		$$ y_2 = y_1 \int \frac{1}{y_1^2} e^{- \int P(x) \dif x} \dif x $$
	\end{proof}

\subsubsection{Nehomogena linearna diferencijalna jednačina drugog reda}
$$ L^n(y) = F(x) $$
U slučaju $n = 2$ snižavanje reda pomožu jednog partikularnog rešenja dovodi do linearne nehomogene jednačine prvog reda.
\begin{theorem}
	Neka je $y_h$ opšte ređenjee homogene jednačine $L^n(y) = 0$ i $y_p$ jedno partikularno rešenje nehomogene jednačine $L^n(y) = F(x)$. Tada je opšte rešenje nehomogene jednačine $L^n(y)=F(x)$ dato sa 
	$$y = y_h + y_p$$
\end{theorem}
\begin{proof}
	Neposredno se proverava da je $L^n(y_h + y_p) = L^n(y_h) + L^n(y_p) = 0 + F(x) = F(x)$ jer je $F(x) = L^n(y_p)$ i $0 = L^n(y_h)$
\end{proof}
\begin{theorem}
	\textbf{Metod varijacije konstanti (Lagranžov):} Neka je $y_h = c_1 y_1(x) + c_2 y_2(x) + \ldots + c_n y_n(x)$ opšte rešenje homogene jednačine $L^n(y) = 0$ tada je opšte rešenje nehomogene jednačine $L^n(y) = F(x)$ dato sa $y = c_1(x) y_1(x) + c_2(x) y_2(x) + \ldots + c_n (x) y_n(x)$ gde su $c_1(x), \, c_2(x), \, \ldots, \, c_n(x)$ funkcije čiji se izvodi nalaze rešavanjem sistema jednačina:
	\begin{gather*}
		c'_1(x) y_1(x) + c'_2(x) y_2(x) + \ldots + c'_n (x) y_n(x) = 0\\
		c'_1(x) y'_1(x) + c'_2(x) y'_2(x) + \ldots + c'_n (x) y'_n(x) = 0\\
		\vdots\\
		c'_1(x) y_1^{(n)}(x) + c'_2(x) y^{(n)}_2(x) + \ldots + c'_n (x) y^{(n)}_n(x) = F(x)\\
	\end{gather*}
	a zatim se same funkcije $c_1(x), \, c_2(x), \, \ldots, \, c_n(x)$ nalaze integracijom:
	\begin{gather*}
		\int c'_1(x) \dif = c_1(x) + c^*_1\\
		\int c'_2(x) \dif = c_2(x) + c^*_2\\
		\vdots\\
		\int c'_n(x) \dif = c_n(x) + c^*_n\\
	\end{gather*}
	Opšte rešenje se dobija kao $y = y_p + y_h$ gde je jedno partikularno rešenje  $y_p = y_1 c_1(x) + y_2 c_2 (x) + \ldots + y_n c_n (x)$.
\end{theorem}

\subsubsection{Linearna diferencijalna jednačina $n$-tog reda sa konstantnim koeficijentima}
\begin{definition}
	\underline{Homogena} jednačina $L^n(y) = y^{(n)} + p_1 y^{(n-1)} + \ldots + p_n y = 0$ gde su $p_i \in \overline{\realset}$ za $i = \overline{1, n}$ konstante, naziva se \textbf{\underline{homogena} linearna diferencijalna jednačina $n$-tog reda sa konstantnim koeficijentima}\\
\end{definition}
Ovaj tip diferencijalnih jednačina rešava se traženjem partikularnog rešenja u obliku $y = e^{\lambda x}$. Zamenom u jednačinu dobijamo $e^{\lambda x}(\lambda^n + p_1 \lambda^{n-1} + \ldots + p_n) = 0$
\begin{definition}
	Algebarska jednačina $\lambda^n + p_1 \lambda^{n-1} + \ldots + p_{n-1} \lambda + p_n = 0$ zove se karakteristična jednačina prethodno navedene diferencijalne jednačine.
\end{definition}
Karakteristična jednačina ima $n$ korena i svakom korenu odgovara po jedno partikularno rešenje, po sledećim pravilima:
\begin{enumerate}[label = \arabic*)]
	\item Svakom realnom, jednostavnom korenu $\lambda$ odgovara jedno partikularno rešenje $y_p = e^{\lambda x}$
	\item Svakom realnom korenu $\lambda$ kada $k > 1$ odgovara $k$ partikularnih rešenja $e^{\lambda x}, \, x e^{\lambda x}, \, \ldots, \, x^{k-1} e^{\lambda x}$
	\item Svakom paru kompleksnih, jednostavnih korena $\alpha \pm i \beta$ odgovaraju partikularna rešenja $e^{\alpha x}\cos \beta x, \, e^{\alpha x} \sin \beta x$.
	\item Svakom paru kompleksnih korena $\alpha \pm i \beta$ kada $k > 1$ odgovara $2k$ partikularnih rešenja:
	\begin{gather*}
		e^{\alpha x} \cos \beta x, \, xe^{\alpha x} \cos \beta x, \, \ldots, \, x^{k-1}e^{\alpha x} \cos \beta x, \, \\
		e^{\alpha x} \sin \beta x, \, xe^{\alpha x} \sin \beta x, \, \ldots, \, x^{k-1}e^{\alpha x} \sin \beta x, \, \\
	\end{gather*}
\end{enumerate}
Primenom ovih pravila na sve korene karakteristične jednačine dobija se skup od ukupno $n$ linearno nezavisnih rešenja $y_1, y_2, \ldots, y_n$ diferencijalne jednačine pa je opšte rešenje homogene diferencijalne jednačine $y_h = c_1 y_1(x) + c_2 y_2(x) + \ldots + c_n y_n(x)$
\begin{definition}
	\underline{Nehomogena} jednačina $L^n(y) = y^{(n)} + p_1 y^{(n-1)} + \ldots + p_n y = F(x)$ gde su $p_i \in \overline{\realset}$ za $i = \overline{1, n}$ konstante, naziva se \textbf{\underline{nehomogena} linearna diferencijalna jednačina $n$-tog reda sa konstantnim koeficijentima}
\end{definition}
Rešavanje nehomogene jednačine sastoji se iz 2 koraka:
\begin{enumerate}[label = \arabic*)]
	\item Reši se homogena jednačina (odredi se $y_h$)
	\item Određuje se opšte rešenje nehomogene jednačine metodom varijacije konstantni
\end{enumerate}
\Napomena ako je funkcija $F(x)$ specijalnog oblika (izvodi su istog oblika kao sama funkcija - polinomska/eksponencijalna/trigonometrijska funkcija, ili su zbirovi/proizvodi tih funkcija), partikularno rešenje $y_p$ može se odrediti metodom neodređenih koeficijenata (više o tome u 3.4.5) tada je opšte rešenje $y = y_h + y_p$
\subsubsection{Metod neodređenih koeficijenata}
	Metod neodređenih koeficijenata sastoji se u tome da se partikularno rešenje nehomogene jednačine pretpostavi u obliku koji je sličan obliku funkcije $F(x)$ ali sa neodređenim koeficijentima koji se određuju zamenom u diferencijalnu jednačinu. Kada se ovako nađe partikularno rešenje $y_p$, onda je opšte rešenje $y = y_h + y_p$.
Ovaj metod primenjuje se samo u slučaju kada je
:
\begin{enumerate}[label = \arabic*)]
	\item 
		$F(x) = e^{\alpha x} Pm(x)$ gde je $Pm(x)$ - polinom stepena $m$
		\begin{enumerate}[label = \alph*)]
			\item  $\alpha$ nije koren karakteristične jednačine $y_p = e^{\alpha x} Qm(x)$
			\item  $\alpha$ je koren reda $k$, $k \geq 1$ karakteristične jednačine $y_p = x^k e^{\alpha x} Qm(x)$
		\end{enumerate}
		Koeficijenti polinoma $Qm(x)$ nalaze se zamenom u nehomogenu diferencijalnu jednačinu, metodom neofređenih koeficijenata.
	
	\item 
		$F(x) = e^{\alpha x} (Pm_1(x)\cos \beta x + Pm_2???(x)\sin \beta x)$
		\begin{enumerate}[label = \alph*)]
			\item $\alpha \pm i \beta$ nije koren karakteristične jednačine $y_p = e^{\alpha x} (Qm(x)  \cos \beta x + Rm(x) \sin \beta x) \quad dgQm dgRm = max{m_1, m_2}$
			\item  $\alpha \pm i \beta$ je koren reda $k$, $k \geq 1$ karakteristične jednačine $y_p = x^k e^{\alpha x} (Qm(x)  \cos \beta x + Rm(x) \sin \beta x) \quad dgQm dgRm = max{m_1, m_2}$
		\end{enumerate}
		koeficijenti polinoma $Qm(x), \, Rm(x)$ nalaze se zamenom u nehomogenu diferencijalnu jednačinu metodom neodređenih koeficijenata
\end{enumerate}